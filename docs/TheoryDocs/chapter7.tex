\chapter{Moment Method Formulation}
\label{chap:mom}
In this chapter we describe a procedure for determining
the induced electric or magnetic surface currents on a single-sheet FSS
(frequency selective surface) located at one of the junction planes in
a multiply stratified medium, when excited by an incident plane wave.  
The analysis is performed using the method of moments, employing
a space-domain formulation of the potential Green's functions, 
which are  described in previous chapters.
Once the equivalent induced
surface currents have been determined, the single-sheet scattering
parameters can be extracted using the formulas presented in
Chapter~\ref{chap:incgsm}.  Scattering parameters of more complicated structures
consisting of several cascaded FSS sheets interlaced with dielectric
layers can be determined from the individual sheet scattering parameters using
the results of Chapter~\ref{chap:gsm}.
%
The following novel features are incorporated in this analysis:
\begin{itemize}
\item The use of a wide-band expansion of the stratified medium
  periodic Green's functions has been incorporated into the moment
  method procedure, greatly reducing the time needed to compute the
  elements of the interaction matrices.  This technique was first
  reported in \cite{simo:01b}.
\item Modifications have been introduced into the triangle subdomain
  basis functions  of Rao, Wilton, and Glisson \cite{rawg:82} to
  enable representation of currents that cross unit cell boundaries.
  This work was previously reported in \cite{simo:02c}.
\end{itemize}
%
%
\begin{figure}[tbhp]
  \begin{center} \footnotesize
    \leavevmode
    \pspicture(1,4.2)(12.9,10)
    \psset{nodesep=2.5pt}
                                % Begin Input Medium:
    \rput(1.5,7){\cstack{Layer \\1 \\ $(\mu_1,\epsilon_1)$}}
    \psline[linewidth=1pt](2.3,4.5)(2.3,8.5)  \uput[d](2.3,4.5){$z_1$}
    \pnode(2.3,8.5){Junction1}
    \rput[b](1.8,9.5){\rnode{Junction1label}{Junction~1}}
    \ncline[linewidth=0.5pt]{->}{Junction1label}{Junction1}
                                % Begin Layer 2:
    \rput(3.15,7){\cstack{Layer \\ 2 \\ $(\mu_2,\epsilon_2)$}}
    \psline[linewidth=1pt](4,4.5)(4,8.5)    \uput[d](4,4.5){$z_2$}
    \pnode(4,8.5){Junction2}
    \rput[b](4,9.5){\rnode{Junction2label}{Junction 2}}
    \ncline[linewidth=0.5pt]{->}{Junction2label}{Junction2}
    \pcline[linewidth=0.5pt,nodesep=1pt]{<->}(2.3,5.5)(4,5.5) \lput*{:U}{$h^{(2)}$}
                                % Begin Layer 3:
    \rput(5,7){\cstack{Layer \\ 3 \\ $(\mu_3,\epsilon_3)$}}
    \psline[linewidth=1pt](6,4.5)(6,8.5)    \uput[d](6,4.5){$z_3$}
    \pnode(6,8.5){Junction3}
    \rput[b](5.5,9){\rnode{Junction3label}{Junction 3}}
    \ncline[linewidth=0.5pt]{->}{Junction3label}{Junction3}
    \pcline[linewidth=0.5pt,nodesep=1pt]{<->}(4,5.5)(6,5.5) \lput*{:U}{$h^{(3)}$}
                                % Ellipses:
    \rput(6.5,6.5){\large$\boldsymbol{\cdots}$}
                                % Begin junction N-3:
    \psline[linewidth=1pt](7,4.5)(7,8.5)    \uput[d](7,4.5){$z_{N-3}$}
    \pnode(7,8.5){JunctionNsm2}
    \rput[b](6.5,9.5){\rnode{JunctionNsm2label}{Junction $N-3$}}
    \ncline[linewidth=0.5pt]{->}{JunctionNsm2label}{JunctionNsm2}
                                % Begin Layer N-2:
    \rput(8.125,7){\cstack{Layer \\ $N-2$ \\ $(\mu_{N-2},\epsilon_{N-2})$}}
    \psline[linewidth=1pt](9.25,4.5)(9.25,8.5)    \uput[d](9.25,4.5){$z_{N-2}$}  
    \pnode(9.25,8.5){JunctionNsm1}
    \rput[b](9.25,9){\rnode{JunctionNsm1label}{Junction $N-2$}}
    \ncline[linewidth=0.5pt]{->}{JunctionNsm1label}{JunctionNsm1}
    \pcline[linewidth=0.5pt,nodesep=1pt]{<->}(7,5.5)(9.25,5.5) \lput*{:U}{$h^{(N-2)}$}
                                % Begin Layer N-1:
    \rput(10.375,7){\cstack{Layer \\ $N-1$ \\ $(\mu_{N-1},\epsilon_{N-1})$}}
    \psline[linewidth=1pt](11.5,4.5)(11.5,8.5)    \uput[d](11.5,4.5){$z_{N-1}$}    
    \pnode(11.5,8.5){JunctionNs}
    \rput[b](11.5,9.5){\rnode{JunctionNslabel}{Junction $N-1$}}
    \ncline[linewidth=0.5pt]{->}{JunctionNslabel}{JunctionNs}
    \pcline[linewidth=0.5pt,nodesep=1pt]{<->}(9.25,5.5)(11.5,5.5) \lput*{:U}{$h^{(N-1)}$}
                                % Begin Output Medium:
    \rput(12.4,7){\cstack{Layer \\ $N$ \\ $(\mu_{N},\epsilon_{N})$}}
    \endpspicture
  \end{center}
  \caption{The structure under consideration is a stack of $N \geq 2$
    dielectric layers.  Layers~1 and $N$ are semi-infinite in extent.
    For $2 \leq i \leq N-1$, layer~$i$ lies in the region $z_{i-1} < z
    < z_i$, is of thickness $h\eye = z_i - z_{i-1}$, and is characterized by
    permeability $\mu_i$ and permittivity $\epsilon_i$.}
  \label{fig:geom4}
\end{figure}
%

The FSS is located in the interface plane $z=z_s$ of the multiple
layered structure shown in Figure~\ref{fig:geom4}.
The structure is laterally invariant, with each dielectric layer being
homogeneous and isotropic.
Each layer $i=1,2, \ldots, N$ is characterized by a complex
permittivity $\epsilon_i$ and permeability $\mu_i$, each of which lies
either in the fourth quadrant of the complex plane, or on the real axis.  The
medium intrinsic wavenumbers are $k_i = \omega\sqrt{\mu_i\epsilon_i}$.
We will assume that the single FSS sheet is located at interface
number $s$, located between layers $s$ and $s+1$, where $1 \leq s < N.$


In general, we can model a zero-thickness, perfectly conducting FSS
using either electric or magnetic currents as the unknowns.  In the
former case, the metalization is removed, and the support\footnote{%
  The support of a function is defined as the closure of the set of
  points where the function takes nonzero values.}%
of the unknown electric
surface current consists of the region previously occupied by the
metalization.  In the latter case, the entire interface plane $z=0$ is
replaced with a zero-thickness PEC (perfect electric conductor) and
the support of the magnetic current is the portion of the plane
formerly {\em not} occupied by the PEC.  

One generally chooses the type of equivalent currents employed in a
given problem so as to minimize the area of the currents' support, and
thus the number of unknowns to be solved for in the MoM problem.
For aperture-type elements magnetic currents are usually
selected; for wire-type elements electric currents are the natural choice.
In the case where an FSS is constructed of lossy material, we are forced
to use electric currents as the unknowns.

\section{Electric Current Mixed Potential Integral Equation}
In this section we consider an FSS that is modeled using equivalent
electric surface current.  This choice is convenient when less than
half of the unit cell area is occupied by metal or when the FSS is
etched onto a sheet of lossy material.

The boundary value problem to be solved is \cite{mitz:68}
\begin{equation}
  Z_s(\vecrho) \Js(\vecrho)  -\Idemfactor_z \bdot \E\scat(\vecrho) = 
  \Idemfactor_z \bdot \E\inc(\vecrho), \quad (\vecrho \in S)
  \label{eq:EFIE}
\end{equation}
where $\E\scat$ is the scattered electric field due to the equivalent
currents flowing 
on the surface $S$ of the FSS, $\E\inc$ is the incident electric field
(the plane wave, including incident, reflected, and transmitted
components, that would exist in the absence of the FSS), $Z_s$ is the
surface resistance of the FSS sheet material,
and $\Idemfactor_z = \x\x + \y\y$ is the unit surface dyadic.  In
terms of potentials this becomes
\begin{equation}
  Z_s(\vecrho) \Js(\vecrho) + [j\omega \A(\vecrho) +
  \gradient\Phi(\vecrho)] \bdot \Idemfactor_z = 
  \E\inc(\vecrho) \bdot \Idemfactor_z, \quad (\vecrho \in S)
  \label{eq:MPIE}
\end{equation}
where the potentials can be expressed as superposition integrals
using appropriately defined Green's functions 
(Chapter~\ref{chap:gfstratified})
$\GAxx$ and $\GPhi$:
\begin{gather}
  \Idemfactor_z \bdot \A(\vecrho) = 
  \iint \GAxx(\vecrho-\vecrho') \cdot \Js(\vecrho') \, \d S',
  \label{eq:Asup}\\
  \Phi(\vecrho) = \iint \GPhi(\vecrho-\vecrho') \cdot \qe(\vecrho') \, \d S'.
  \label{eq:Phisup}
\end{gather}
In \eqref{eq:Asup} and \eqref{eq:Phisup}, $\Js$ and $\qe$ are the unknown
induced electric surface current and surface charge densities,
respectively, which will be determined
by enforcing
Equation~\eqref{eq:MPIE} in an approximate manner using the method of
moments.  We use the triangle subdomain basis functions of Rao, Wilton and
Glisson \cite{rawg:82}, suitably modified so as to accommodate the
periodic boundary conditions encountered in the unit cell analysis.


\subsection{Basis Functions}
\label{sec:bf}
We now enunciate a few important definitions and 
and properties of the basis functions in order to establish
notation.
The support of $\Js$ is first partitioned into a number of triangles.
In \cite{rawg:82} a basis function is defined over each pair of triangles which share a common
edge.  In this work, we include not only these adjacent pairs of
triangles, but also those pairs of triangles which would be adjacent
if one of the pair were translated by $\s_1$ or $\s_2$ from its actual
position.

Consider first a typical pair of adjacent triangles; their common edge
is not on the boundary of the unit cell.
Figure~\ref{fig:basis} shows two such triangles, $T_m^+$ and $T_m^-$,
which comprise the support of the $m$th basis function and which
share an interior edge of the triangulated surface.  Points in $T_m^+$ may be designated 
by either the position vector $\r$ which locates them with respect to the global
origin, or by $\vecrho_m^+$, which is defined with respect to the free vertex
of $T_m^+$.  The vector $\vecrho_m^-$ is defined similarly for points in $T_m^-$, 
except that it is directed {\em towards} the free vertex of $T_m^-$.
Which of the two triangles is designated ``plus'' and which is ``minus'' depends
on an arbitrary choice of positive current reference direction for the
$m$th basis function.
Positive reference current flows across the edge from $T_m^+$ to $T_m^-$. The basis
function associated with the $m$th edge is then defined as
\begin{equation}
  \label{eq:basis}
  \f_m(\r) = 
  \begin{cases}
    \displaystyle
    \frac{l_m e^{j\theta_m^+}}{2A_m^+} \vecrho_m^+ & \text{if $\r \in T_m^+$} \\
    \displaystyle
    \frac{\rule{0pt}{3ex}l_m e^{j\theta_m^-}}{2A_m^-} \vecrho_m^- & \text{if $\r \in T_m^-$} \\
    \0 & \text{otherwise},
  \end{cases}
\end{equation}
where $l_m$ is the length of the common edge, $A_m^{\pm}$ is the area
of triangle $T_m^{\pm}$, and $\theta_m^+ = \theta_m^- = 0$ in this case.
%
\begin{figure}[tbp]
  \begin{center}
    \leavevmode
      \pspicture(-5,-2)(5,5)
      % Begin by defining locations of triangle vertices as PStricks nodes:
      \pnode(-5,0){Pfree}  % The free node of the plus triangle.
      \pnode(4.5,2){Mfree}  % The free node of the plus triangle.
      \pnode(1,4.5){Top}     % The top node of the common edge.
      \pnode(1,0){Bot}     % The bottom node of the common edge.
      \rput(0.5,-2){\rnode{Origin}{$\bullet$}} % The coordinate system origin.
      \rput[l](0.5,-2){\quad Origin}
        \psset{linewidth=1.3pt}   % For triangle legs
       \psset{arrowsize=6pt,arrowinset=0.5,arrowlength=2.5}
        \ncline{Pfree}{Bot}     % BL triangle leg
        \mput{\rnode{BLEdgeCenter}{}}  % Put node at center of bottom left edge.
        \ncline{Pfree}{Top}     % TL triangle leg
        \aput(0.5){$T_m^+$}
        \ncline{Mfree}{Bot}     % BR triangle edge
        \ncline{Mfree}{Top}     % TR triangle edge
        \bput(0.5){$T_m^-$}
        \ncline{Top}{Bot}       % Common triangle edge
        \mput{\rnode{CommonEdgeCenter}{}}  % Put node at center of common edge.
        \psset{linestyle=none}   % Don't want to see the next few lines.
        \ncline{Pfree}{CommonEdgeCenter} 
        \lput(0.6667){\rnode{Pcentroid}{\Large$\bullet$}}  % Positive triangle centroid
        \ncline{Mfree}{CommonEdgeCenter} 
        \lput(0.6667){\rnode{Mcentroid}{\Large$\bullet$}}  % Negative triangle centroid
        \ncline{Pcentroid}{BLEdgeCenter}
        \lput(0.4){\rnode{ObsPoint}{\Large$\bullet$}}  % Observation point location.
        %
        \psset{linestyle=solid,linewidth=1pt,nodesepA=0pt,nodesepB=0pt} 
        \ncline{->}{Pfree}{Pcentroid}  \aput[1pt](0.7){$\vecrhocp_m$}
        \ncline{->}{Pfree}{ObsPoint}  \bput[1pt](0.7){$\vecrho^+_m$}
        \ncline{->}{Mcentroid}{Mfree}  \aput[2pt](0.5){$\vecrhocm_m$}
                                %
        \ncline{->}{Origin}{Pcentroid}  \bput[1pt](0.4){$\rcp_m$}
        \ncline{->}{Origin}{Mcentroid}  \bput[1pt](0.3){$\rcm_m$}
        \ncline{->}{Origin}{ObsPoint}   \aput(0.5){$\r$}
        \ncline[arrowsize=3pt 2,arrowinset=0.5,arrowlength=2,%
        linewidth=0.6pt,offset=10pt,nodesep=1pt]{<->}{Bot}{Top}
        \ncline[linewidth=0.6pt,offset=10pt,nodesep=0pt]{|-|}{Bot}{Top} \lput*{:U}{$l_m$}
      \endpspicture
    \caption{Triangular basis function geometry showing two triangles with
      a common edge.  The superscript ``c'' denotes the centroid of the triangle.}
    \label{fig:basis}
  \end{center}
\end{figure}
%
A few facts about the basis functions should be kept in mind.  First, 
the basis functions are unitless.  Units are carried by the expansion
coefficients associated with the basis functions.  Second,
the 
support of a single basis function is limited to two, typically adjacent triangles.
Third, the total current density in any given triangle is the vector sum
of contributions from up to three distinct basis functions whose common support
includes that triangle.  Fourth, any current flow normal to an edge is due 
entirely to the basis function associated with that edge.  Finally, the
basis functions are normalized so that the normal component of current
density crossing the defining edge is unity%
%
\footnote{It is easy to see that the normal current density crossing
  an edge is constant by
recalling that the equation of a plane in space is $\n \bdot \r = \text{constant}$.
Similarly, the equation of the line containing the defining edge for the $m$th
basis function is $\n \bdot \vecrho_m^{\pm} = \text{constant}$.  In fact, the
normalization for the basis functions is selected so that this constant is
unity in magnitude.}
%
at any point of the edge.
This means that the expansion
coefficients $\{\Icoef_m\}$ defined below can be interpreted as the
total current
crossing the associated edge(s) of the triangulated surface.


This definition is identical to that of \cite{rawg:82} except for the
introduction of the factors containing $\theta_n^\pm$.  To see why
these are necessary, consider the situation shown in
Figure~\ref{fig:basis2}.  Points within the unit cell are
parameterized using
\begin{equation}
  \vecrho = \xi \s_1 + \eta \s_2, \quad 0 \leq \xi < 1, \; 0 \leq \eta
  < 1.
\end{equation}
%
\begin{figure}[tbp]
  \begin{center}
    \leavevmode
      \psset{unit=0.3in,linewidth=0.8pt}
      \pspicture(0,0)(10,10)
      % Begin by defining locations of triangle vertices as PStricks nodes:
      \pnode(0,0){Origin}  % The origin is at the lower left corner
      \pnode(9,3){S1}
      \pnode(1,7){S2}
      \pnode(10,10){S1plusS2}
      \pnode(2,3){TplusC}  % Third node of plus triangle
      \pnode(6.5,4){TminusC}  % Third node of plus triangle
      %  Now draw vectors bounding the unit cell:
      \ncline[arrowsize=6pt,arrowinset=0.5,arrowlength=2.5]{->}{Origin}{S1} 
      \bput(0.99){$\s_1$}  \bput{:U}(0.5){$\eta = 0$}
      \ncline[arrowsize=6pt,arrowinset=0.5,arrowlength=2.5]{->}{Origin}{S2} 
      \aput(0.99){$\s_2$} \aput{:U}(0.6){$\xi = 0$}
      \lput(0.1){\rnode{TplusA}{}}  % Define first node of plus triangle
      \lput(0.4){\rnode{TplusB}{}}  % Define second node of plus triangle
      \ncline{S1}{S1plusS2} 
      \bput{:U}(0.6){$\xi = 1$}
      \lput(0.1){\rnode{TminusA}{}}  % Define first node of minus triangle
      \lput(0.4){\rnode{TminusB}{}}  % Define second node of minus triangle
      \ncline{S2}{S1plusS2} 
       \aput{:U}(0.5){$\eta = 1$}
       %
       %  Draw the ``plus'' triangle at xi = 0:
       {\psset{linewidth=1.5pt}
         \ncline{TplusA}{TplusB} \ncline{TplusB}{TplusC}
         \ncline{TplusC}{TplusA} }
       \aput[1pt](0.3){$T_m^+$}
       %  Draw the ``minus'' triangle at xi = 1:
       {\psset{linewidth=1.5pt}
       \ncline{TminusA}{TminusB} \ncline{TminusB}{TminusC} \ncline{TminusC}{TminusA} }
       \bput[1pt](0.2){$T_m^-$}
       % Draw the constant eta lines bounding the triangle edges at
       % the unit cell boundaries:
       \ncline[linestyle=dashed]{TplusA}{TminusA}
       \aput{:U}{$\eta=\eta_1$}
       \ncline[linestyle=dashed]{TplusB}{TminusB}
       \aput{:U}{$\eta=\eta_2$}
       % Draw the dimensioning line at xi = 0:
       \ncline[arrowsize=3pt 2,arrowinset=0.5,arrowlength=2,%
        linewidth=0.6pt,offset=9pt,nodesep=1pt]{<->}{TplusA}{TplusB}
        \ncline[linewidth=0.6pt,offset=9pt,nodesep=0pt]{|-|}{TplusA}{TplusB} 
        \lput*{:U}{$l_m$}
       % Draw the dimensioning line at xi = 1:
       \ncline[arrowsize=3pt 2,arrowinset=0.5,arrowlength=2,%
        linewidth=0.6pt,offset=-9pt,nodesep=1pt]{<->}{TminusA}{TminusB}
        \ncline[linewidth=0.6pt,offset=-9pt,nodesep=0pt]{|-|}{TminusA}{TminusB} 
        \lput*{:U}{$l_m$}
      \endpspicture
    \caption{Triangular basis function geometry showing a pair of triangles
      located at the $\xi = \text{constant}$ unit cell boundaries.}
    \label{fig:basis2}
  \end{center}
\end{figure}
%
Unit cell boundaries are located at $\xi = 0$, $\xi = 1$, $\eta = 0$,
and $\eta = 1$.  In order to preserve the periodicity of the computed
currents we agree to triangulate the unit cell in such a way
that the number and location of the resulting edges along the $\xi = 0$ and $\xi
= 1$ boundaries are identical, and similarly for the $\eta = 0$ and $\eta
= 1$ boundaries.  A pair of triangles $T_m^+$ and $T_m^-$ with 
edges at the $\xi=0$ and
$\xi=1$ boundaries, respectively, are shown in the figure.  
These edges both span the same range of $\eta$ and so are parallel and
congruent. A basis function is defined for each such pair of triangles
on the $\xi = \text{constant}$ or $\eta = \text{constant}$ boundaries.
We focus attention upon the basis function $\f_m$ whose support is
the union of the two triangle faces shown in the figure.  
Because of the boundary condition
\begin{equation}
  V(\r + m\s_1 + n\s_2) = V(\r) e^{-j(m\psi_1 + n \psi_2)}, \quad
  \forall m,n \in \Integers
\end{equation}
enforced on all currents and fields in the unit cell,
it must be true that the normal current density crossing the edge at $\xi=1$
is equal to that crossing the $\xi=0$ edge multiplied by the factor
$e^{-j\psi_1}$.  Therefore, we must insist that
\begin{equation}
  \theta_m^- = \theta_m^+ - \psi_1.
\end{equation}
We will establish the convention that $\theta_m^\pm = 0$ for all edges
except those along the $\xi=1$ and $\eta=1$ unit cell boundaries.  Therefore, for
the situation shown in Figure~\ref{fig:basis2} we have $\theta_m^- =
-\psi_1$ and similarly for all other triangles with one edge located
on the $\xi = 1$ boundary.  For triangles with an edge on the $\eta=1$
boundary we set the corresponding phase to $-\psi_2$.  

We now expand the unknown electric surface current density in a series of these basis
functions:
\begin{equation}
  \label{eq:Jexpand}
  \Js(\vecrho) =  \sum_{n=1}^{N_J} \frac{\Icoef_n}{l_n} \f_n(\vecrho).
\end{equation}
The number of basis functions
$N_J$ is equal to the number of interior edges of the triangulated surface in the
$z=0$ plane of the unit cell plus the number of edges along the $\xi=0$ and $\eta=0$
borders of the unit cell.  Note that $\Icoef_n$ has units of \si{A}.


\subsection{Impedance Matrix}
\subsubsection{Reduction to Matrix Equation}
Having expanded the unknown current in a series of suitable basis functions, 
it is now desired to determine the coefficients in a way that satisfies 
Equation~\eqref{eq:MPIE} as closely as possible.  
The method of moments transforms the functional Equation~\eqref{eq:MPIE}
to a matrix equation by enforcing it in an average manner.
This is accomplished by first defining an inner product which maps a pair of
functions to a single complex number, and then repeatedly
{\em testing\/} the equation by taking the inner product of both sides with a 
number of {\em testing\/} or {\em weighting\/} functions.

The inner product used in testing Equation~\eqref{eq:MPIE} is
\begin{equation}
  \innerprod{\f, \boldsymbol{g}} = \iint \f \bdot \boldsymbol{g}^* \; \d S.
\end{equation}
The asterisk denotes complex conjugation and the integration is performed over 
the portion of the $z=0$ plane containing the common support of the two 
functions $\f$ and $\boldsymbol{g}$.
We choose our testing functions to be the same as our basis functions 
(Galerkin's method).  Then taking the inner product of \eqref{eq:MPIE}
with each of the functions $\f_m/l_m$  for $m = 1,2,\ldots, N_J$ yields
\begin{multline}
  j\omega \innerprod{\A,\f_m/l_m} + \innerprod{\gradient\Phi,\f_m/l_m}
  + \innerprod{Z_s \Js, \f_m/l_m} 
          = \innerprod{\E\inc, \f_m/l_m}, \\
      m = 1, 2, \ldots,  N_J.   \label{eq:Emom}
\end{multline}  
Following \cite{rawg:82} we can approximate the second term of 
\eqref{eq:Emom} by
\begin{equation}
  \innerprod{\gradient\Phi, \f_m/l_m} \approx 
         e^{-j\theta_m^-} \Phi(\rcm_m) - e^{-j\theta_m^+} \Phi(\rcp_m)  \label{eq:Phimom}
\end{equation}
where $\rcpm_m$ is the centroid of triangle $T_m^{\pm}$. The first
term is also approximated as in the reference:
\begin{equation}
  \innerprod{\A, \f_m/l_m} \approx \frac{1}{2} 
      \left[ \A(\rcp_m) \bdot \vecrhocp_m e^{-j\theta_m^+} +  
             \A(\rcm_m) \bdot \vecrhocm_m  e^{-j\theta_m^-}
      \right].      \label{eq:Amom}
\end{equation}  
$\vecrhocpm_m$ is the vector $\vecrho_m^{\pm}$ evaluated at the centroid
of the corresponding triangle.


Substituting the expansion of the current \eqref{eq:Jexpand} into \eqref{eq:Emom}
yields the desired matrix equation.
\begin{equation}
  \mat{Z} \mat{I} =  \mat{V},  \label{eq:matrix}
\end{equation}
where $\mat{Z} = [\matel{Z}_{mn}]$ is the so-called {\em generalized impedance 
matrix}, a square matrix of order $N_J$ with units of \si{\ohm}.  
$\mat{I} = [\Icoef_m]$ is the column matrix of unknown current
coefficients with units of $\si{A}$. 
$\mat{V}= [\matel{V}_m]$ is the excitation vector (or {\em generalized
voltage vector}) with units of $\si{V}$.
The elements of the generalized impedance matrix are
\begin{multline}
  \matel{Z}_{mn} = 
  \innerprod{Z_s \f_n/l_n,\f_m/l_m}
   \\
   \mbox{} + j \omega \left(
    \A_{mn}^+ \bdot \frac{\vecrhocp_m}{2} e^{-j\theta_m^+}
    +
    \A_{mn}^- \bdot \frac{\vecrhocm_m}{2} e^{-j\theta_m^-}
  \right)
  + \Phi_{mn}^- e^{-j\theta_m^-} - \Phi_{mn}^+ e^{-j\theta_m^+}
\end{multline}
where
\begin{align}
  \A_{mn}^{\pm} & \equiv 
  \frac{1}{l_n} \iint  
  \GAxx(\rcpm_m;\r') \, \f_n(\r') \; \d S'  \notag \\
  &= 
    \frac{e^{j\theta_n^+}}{2A_n^+} \iint_{T_n^+}  \!\!\!
    \GAxx(\rcpm_m;\r') \, \vecrho_n^+ \; \d S'
    +
    \frac{e^{j\theta_n^-}}{2A_n^-} \iint_{T_n^-}  \!\!\!
    \GAxx(\rcpm_m;\r')  \, \vecrho_n^- \; \d S'
    \label{eq:Amnpm}
\end{align}
and
\begin{align}
  \Phi_{mn}^{\pm} &\equiv \frac{j}{\omega l_n} \iint
  \GPhi(\rcpm_m;\r')
  \gradient'_s \bdot \f_n(\r')  
  \; \d S'
  \notag \\
  &= \frac{j}{\omega}
  \left[
    \frac{e^{j\theta_n^+}}{A_n^+} \iint_{T_n^+}  \!\!\!
    \GPhi(\rcpm_m;\r') \; \d S'
    -
    \frac{e^{j\theta_n^-}}{A_n^-} \iint_{T_n^-}{}  \!\!\!
    \GPhi(\rcpm_m;\r') \; \d S'
  \right]    \label{eq:Phimnpm}
\end{align}
and we have used the fact \cite{rawg:82} that
\begin{equation}  
  \gradient_s \bdot \f_n(\r) = 
  \begin{cases}
    \displaystyle  \frac{l_n}{A_n^+} e^{j\theta_n^+}  & \text{if $\r \in T_n^+$} \\
    \displaystyle \frac{\rule{0pt}{2ex}-l_n}{A_n^-} 
    e^{j\theta_n^-} & \text{if $\r \in T_n^-$} \\
    0                               & \text{otherwise}
  \end{cases}
  \label{eq:fdiv}
\end{equation}    
which holds true for a basis function consisting of a pair of adjacent
triangles.  For basis functions defined on nonadjacent pairs, the
expression for the surface divergence includes a pair of line
singularities whose contributions to \eqref{eq:Phimnpm} can be
shown to cancel.

 \subsubsection{Efficient Evaluation of the Generalized Impedance Matrix}
 The authors of \cite{rawg:82} discuss the way in which redundant calculations can 
 be avoided by considering face-pair contributions to $\mat{Z}$ rather than the 
 edge-pair contributions used in the definitions of \eqref{eq:Amnpm} and
 \eqref{eq:Phimnpm}.  Here we apply the same technique using our
 potential Green's functions specific to this problem.  In addition,
 we individually track the contributions to the impedance matrix 
 exhibiting distinct frequency dependence, thus allowing a
 wide-band implementation.

 \paragraph{Vector Potential Contributions}

 Consider first a typical vector integral needed
 to evaluate $\matel{Z}_{mn}$.  
 The quantity
 \begin{equation}
   \A_i^{uv} =   \frac{e^{j\theta_i^\pm}}{2A^v}  \iint_{T^v}  \!\!\!
           \GAxx(\rcu;\r') \, \vecrho_i \; \d S' \label{eq:Aiuv}
 \end{equation}
 represents the magnetic vector potential at the centroid $\rcu$ of triangle
 $T^u$ due to the $i\text{th}$ basis function defined on triangle $T^v$.
 In \eqref{eq:Aiuv} we have temporarily assumed a local numbering scheme
 on the source triangle, with vertices at $\r_1$, $\r_2$, and $\r_3$,
 and where 
 \begin{equation}
   \vecrho_i =  \pm (\r' - \r_i), \quad i = 1, 2, 3.
   \label{eq:vecrhosubi}
 \end{equation}
 The plus sign is selected in \eqref{eq:Aiuv} and \eqref{eq:vecrhosubi} 
 if the associated basis function defines 
 positive current to flow {\em out\/} through the $i\text{th}$ edge of
 the triangle, which lies opposite the $i$th vertex.
 Using the expressions for the magnetic vector potential Green's
 function given in  Chapter~\ref{chap:gfstratified}, Equation~\eqref{eq:Aiuv} can 
 be written as  the sum of a frequency dependent term arising from the modal
 difference series, added to several contributions arising from the
 spatial series:
 \begin{equation}
   \A_i^{uv} = \frac{\tilde{\mu}}{4\pi} e^{j\theta^{\pm}_i} 
   \left[
     4\pi \I_{1i}^{uv}%(\mu_1,\epsilon_1,\mu_2,\epsilon_2) 
     + 
     \frac{1}{2A^v}  \iint_{T^v}  \!
     \frac{\vecrho_i}{\rhocu_{00}} \, \d S' +
     u \J_i^{uv} +
     \frac{c_3(\scriptstyle\mu_s,\epsilon_s,\mu_{s+1},\epsilon_{s+1})}%
     {u} \K_i^{uv} 
   \right] \label{eq:Aiuv2}
 \end{equation}
 where
 \begin{subequations}
   \begin{align}
     \I_{1i}^{uv} &=
     \frac{1}{2A^v}  \iint_{T^v}  \!\!\!
     \Sigma_{M1}(\rcu;\r')
     \, \vecrho_i \; \d S', \\
     %
     \Sigma_{M1}(\rcu;\r') &= 
     \frac{1}{2A} \summn 
     \left[
       \frac{2\Vi\TE(\vecbeta_{mn},z_s,z_s)}{j\omega\tilde{\mu}}
       - \frac{1}{\kappa_{mn}} -
       \frac{c_3({\scriptstyle\mu_s,\epsilon_s,\mu_{s+1},\epsilon_{s+1}})}%
       {\kappa_{mn}^3}
     \right] e^{-j\vecbeta_{mn}\bdot(\rcu-\r')}, \\
                                %
     \J_i^{uv} &=
     \frac{1}{2A^v}  \iint_{T^v}  \!\!\!
     \vecrho_i \,
     \summn  \frac{e^{-u\rhocu_{mn}}-\delta_{m0}\delta_{n0}}{u\rhocu_{mn}}  
     e^{-j(m\psi_1+n\psi_2)} \; \d S', \\
                                %
     \K_i^{uv} &=
     \frac{1}{2A^v}  \iint_{T^v}  \!\!\!
     \vecrho_i \,
     \summn  e^{-u\rhocu_{mn}}
     e^{-j(m\psi_1+n\psi_2)} \; \d S',
   \end{align}
 \end{subequations}
 and $\rhocu_{mn} = \norm{\rcu  - \r' - m\s_1 -n\s_2}$.  Note that the
 only frequency dependence in \eqref{eq:Aiuv2} occurs in the terms
 $\I_{1i}^{uv}$ and $c_3$.  Therefore, the contrubutions due to the
 singular integral, $\J_i^{uv}$ and $\K_i^{uv}$ can be computed once,
stored, and combined appropriately at each new analysis frequency.
Although $\I_{1i}^{uv}$ must be computed anew at each frequency, the modal
difference series occurring in its integrand can be very rapidly evaluated by
simple interpolation into a two-dimensional table obtained from the
FFT (Section~\ref{sec:modalFFT}).

Calculation of the singular integral in \eqref{eq:Aiuv2} is discussed
in Appendix~\ref{app:singint}.  Efficient evaluation of the remaining
integrals is discussed next.

Each of the integrals $\I_{1i}^{uv}$, $\J_i^{uv}$, 
and $\K_i^{uv}$ is of
the form
\begin{equation}
  \L_i = \x L_{ix} + \y L_{iy} = \frac{1}{2A^v} \iint_{T^v} \!\!\!
  \vecrho_i \,
  f(\r;\r') \, \d S'.
\end{equation}
Each involves a bounded, well-behaved
integrand which can be integrated numerically using a low-order Gaussian 
scheme designed especially for surface integration (cubature)
over triangular domains \cite{zien:71}.  However, it is advantageous to first 
express them in terms of integrals which do not depend on the particular basis function
index $i$.  To accomplish 
this, we employ the so-called {\em normalized area coordinates\/} 
$(\xi,\eta,\zeta)$.%
\footnote{In this section ``$\eta$'' is used to mean one of the 
  normalized area coordinates.  Elsewhere in these notes the same symbol 
  may be used to represent the intrinsic impedance of a dielectric medium.}
The source point can be written in terms of area coordinates as
\begin{equation}
  \r' = \xi \r_1 + \eta \r_2 + \zeta \r_3
\end{equation}
so that
\begin{equation}
  \vecrho_i = \pm(\r' - \r_i) = \pm \left(
    \xi \r_1 + \eta \r_2 + \zeta \r_3 - \r_i
  \right).
\end{equation}
where the $\{ \r_i \}$ are the vertices of the source triangle $T^v$ having
area $A^v$. Only two of the three (nonnegative)  
coordinates can be independently specified since $\xi + \eta + \zeta = 1$ for 
points in the triangle. 
The area element in normalized area coordinates is $\d S' = 2 A^v \,\d\xi\,\d\eta$.

The desired integral $\L_{i}$ can
be expressed in terms of scalar integrals over normalized area coordinates 
which are independent of $i$ in the
following way.
\begin{equation}
  \L_{i} = \pm  \left[
    \r_1 L_{\xi} + \r_2 L_{\eta} + \r_3 L_{\zeta}
    - \r_i L
  \right]    \label{eq:Lit}
\end{equation}
where
\begin{subequations}
  \begin{align}
    L_{\xi} &= \int_0^1 \int_0^{1-\eta}  \!\!\!\!
    \xi \, f(\rcu;\xi\r_1 + \eta\r_2 + \zeta\r_3)
    \, \d\xi \, \d\eta   \\
    L_{\eta} &= \int_0^1 \int_0^{1-\eta}  \!\!\!\!
    \eta \, f(\r^{cu};\xi\r_1 + \eta\r_2 + \zeta\r_3)
    \, \d\xi \, \d\eta   \\
    L_{\zeta} &= \int_0^1 \int_0^{1-\eta}  \!\!\!\!
    \zeta \, f(\r^{cu};\xi\r_1 + \eta\r_2 + \zeta\r_3)
    \, \d\xi \, \d\eta   \\
    L &=  \int_0^1 \int_0^{1-\eta}  \!\!\!\!
    f(\r^{cu};\xi\r_1 + \eta\r_2 + \zeta\r_3)
    \, \d\xi \, \d\eta.   \label{eq:L}
  \end{align}
\end{subequations}
It is important to note
that the above integrals are not all independent.  We also have that
\begin{equation}
  L = L_{\xi} + L_{\eta} + L_{\zeta}
\end{equation}
so that in general there are three integrals to be numerically evaluated for
each observation point / source triangle pair.  These integrals,
when weighted according to Equation~\eqref{eq:Lit} contribute
to as many as nine entries of $\mat{Z}$, depending on the number of basis 
functions defined in each triangle.  They are computed using the seven-point
cubature formula found in, e.g., \cite{zien:71} or \cite{bali:84}.
\begin{equation}
    \int_0^1 \int_0^{1-\eta} \!\!\!
  f(\xi, \eta) \; \d\xi \; \d\eta \approx \sum_{k=1}^7 w_k f(\xi_k, \eta_k).
  \label{eq:cubtri}
\end{equation}
The sample points and weights are given in Table~\ref{tab:tri}.
\begin{table}
  \begin{center}
    \begin{tabular}{|c|c|c|c|}  \hline
      {\boldmath $k$\unboldmath\rule[-.5ex]{0ex}{3ex}} &  
      {\boldmath $\xi_k$} \unboldmath  & 
      {\boldmath $\eta_k$ \unboldmath} & {\boldmath $w_k$ \unboldmath} 
      \\ \hline
      1 & 0.33333333333333333 & 0.33333333333333333 & 0.11250000000000000 \\
      2 & 0.10128650732345633 & 0.79742698535308730 & 0.06296959027241358 \\
      3 & 0.79742698535308730 & 0.10128650732345633 & 0.06296959027241358 \\
      4 & 0.10128650732345633 & 0.10128650732345633 & 0.06296959027241358 \\
      5 & 0.47014206410511505 & 0.05971587178976989 & 0.06619707639425308 \\
      6 & 0.05971587178976989 & 0.47014206410511505 & 0.06619707639425308 \\
      7 & 0.47014206410511505 & 0.47014206410511505 & 0.06619707639425308 \\
      \hline 
    \end{tabular}
  \end{center}
  \caption{Sample points and weights for seven point triangular cubature.}
  \label{tab:tri}
\end{table}

\paragraph{Scalar Potential Contributions}
In addition to the vector integrals just discussed, it is also 
necessary to calculate scalar integrals of the form
\begin{equation}
  \Phi_i^{uv} = \pm \frac{je^{j\theta^{\pm}_i}}{\omega A^v} \!
  \iint_{T^v} \!\!
  \GPhi(\r^{cu};\r') \; \d S'.  \label{eq:Phiiuv}
\end{equation}
$\Phi_i^{uv}$ is the electric scalar potential at the centroid of
triangle $T^u$ due to the surface electric charge density associated with
the $i$th basis function defined on triangle $T^v$.  As before, the
sign is taken to be positive if the basis function in question defines
positive reference current to flow {\em out} of $T^v$ through the 
$i$th edge. 
We proceed as before, using the results in Chapter~\ref{chap:gfstratified} for the
electric scalar potential Green's function to write the above integral
as 
\begin{equation}
  \Phi_i^{uv} = \frac{\pm
    je^{j\theta^{\pm}_i}}{2\pi\omega\bar{\epsilon}} 
  \left[
     4\pi I_{2}^{uv} + 
     \frac{1}{2A^v}  \iint_{T^v}  \!
     \frac{\d S'}{\rhocu_{00}} +
     u J^{uv} +
     \frac{d_3(\scriptstyle\mu_s,\epsilon_s,\mu_{s+1},\epsilon_{s+1})}{u} 
     K^{uv}
   \right] \label{eq:Phiiuv2}
\end{equation}
where
\begin{subequations}
  \begin{align}
    I_{2}^{uv} &=
    \frac{1}{2A^v}  \iint_{T^v}  \!\!\!
    \Sigma_{M2}(\rcu;\r')
    \, \d S', \\
                                %
    \Sigma_{M2}(\rcu;\r') & =
    \frac{1}{2A} \summn 
    \left[
      \frac{2j\omega\bar{\epsilon}(\Vi\TM-\Vi\TE)}{\beta_{mn}^2} 
      - \frac{1}{\kappa_{mn}} -
      \frac{d_3({\scriptstyle\mu_s,\epsilon_s,\mu_{s+1},\epsilon_{s+1}})}
      {\kappa_{mn}^3}
    \right]  e^{-j\vecbeta_{mn}\bdot(\rcu-\r')}, \\
                                %
    J^{uv} &=
    \frac{1}{2A^v}  \iint_{T^v}  \!
    \summn  \frac{e^{-u\rhocu_{mn}}-\delta_{m0}\delta_{n0}}{u\rhocu_{mn}}  
    e^{-j(m\psi_1+n\psi_2)} \; \d S', \\
    K^{uv} &=
    \frac{1}{2A^v}  \iint_{T^v}  \!
    \summn  e^{-u\rhocu_{mn}}
    e^{-j(m\psi_1+n\psi_2)} \; \d S'.
  \end{align}
\end{subequations}
Note that $J^{uv}$ and $K^{uv}$ have already been computed as
scalar portions of the corresponding vector integrals 
$\J_{i}^{uv}$ and $\K_{i}^{uv}$, respectively (see Equation~\ref{eq:L}).
The only new quantities to be computed are $I_{2}^{uv}$ and the singular
integral. The integral $I_{2}^{uv}$ can be computed in exactly
the same manner as $I_{1}^{uv}$ using the FFT. Evaluation of the singular 
integral is discussed in Appendix~\ref{app:singint}.
 Note that the
only frequency dependence in \eqref{eq:Phiiuv2} occurs in the 
factor $\omega$ and the terms $I_{2}^{uv}$ and $d_3$. 

\paragraph{Surface Impedance Contributions}
The contribution to the generalized impedance matrix due to finite
surface impedance of the conducting layer is given by the inner
product
\begin{equation}
  \innerprod{Z_s \f_n/l_n,\f_m/l_m}.
\end{equation}
Note that in most cases basis functions $m$ and $n$ do not share any
common support and the inner product yields zero.  
Otherwise, two cases remain. Either
$n=m$, or $n \neq m$ with edges $n$ and $m$  common to a single
triangle. The formulas for the inner product in these cases, assuming that the 
surface impedance is constant on each triangle face,  
are derived in 
Appendix \ref{ap:ip}.  For $n=m$ the result is
\begin{equation}
\innerprod{Z_s \f_n/l_n,\f_n/l_m} = \frac{Z_s^+}{48A_n^+}(3{l_a^+}^2 + 
                     3{l_b^+}^2 - l_n^2)
            + \frac{Z_s^-}{48A_n^-}(3{l_a^-}^2 + 3{l_b^-}^2 - l_n^2)
\end{equation}
where the plus and minus signs refer to triangles $T_n^+$ and $T_n^-$, 
respectively, and $l_a^{\pm}$ and $l_b^{\pm}$ are the lengths of the 
two other edges on these triangles.  For $n \neq m$ the result is
\begin{equation}
\innerprod{Z_s \f_n/l_n,\f_m/l_m} = \pm
e^{j(\theta_n-\theta_m)}
\frac{Z_s^{mn}}{48A^{mn}}
                    (l_m^2 + l_n^2 - 3 l_{p}^2)
\end{equation}
where the double superscript $mn$ refers to the triangle common 
to basis functions $m$ and 
$n$ and $l_{p}$ is the length of the third edge.  The minus sign is taken if 
one of the basis functions $\f_m$ and $\f_n$ define positive current flow out 
of the triangle while the other basis function defines positive current flow 
into the triangle.  If the two basis functions agree in this respect the plus 
sign is used.
  
\subsection{\label{subsec:gvm}Generalized Voltage Vector}
The elements of the generalized voltage vector are defined as 
\begin{equation}
 \matel{V}_m =  \innerprod{\E\inc, \f_m/l_m} = \frac{1}{l_m} 
 \iint \! \E\inc \! \bdot \f_m^* \, \d S, \quad
      m = 1, 2, \ldots,  N_J.  \label{eq:Vm}
\end{equation}
The incident field $\E\inc$ is the field that would exist in the
absence of the FSS metalization due to an incoming, normalized
Floquet mode from Region~$1$ or~$N$ with unit amplitude coefficient.
This scenario was treated in Chapter~\ref{chap:incgsm} and we use the results
derived there.  Let $i \in \{1,N\}$ be the region index designating
the source of the incoming Floquet mode.  The mode index is $q =
(p_q,m_q,n_q)$ where $p_q=1$ for TE modes and $p_q=2$ for TM modes.  The
transverse portion of the incident electric field evaluated in the
$z=z_s$ plane is:
\begin{equation}
  \Idemfactor_z \bdot \E\inc(\vecrho) = V(z_s) \e_q\eye(\vecrho)
  = V(z_s) c_q\eye \t_q e^{-j\vecbeta_{m_q n_q} \bdot
  \vecrho} \label{eq:Eincmom}
\end{equation}
where the modal polarization vector $\t_q$ is defined as
\begin{equation}
  \label{eq:tdef}
  \t_q = 
  \begin{cases}
    \z\cross\betahat_{mn} & p_q = 1 \; \text{(TE modes),} \\
    \betahat_{mn}  & p_q = 2 \; \text{(TM modes),}
  \end{cases}
\end{equation}
and $V(z_s)$ is the equivalent transmission line voltage evaluated at
$z=z_s$ using the formulas presented in Chapter~\ref{chap:incgsm}.

Upon substituting the incident field of \eqref{eq:Eincmom} into
\eqref{eq:Vm} we find that
\begin{align}
  \matel{V}_m &=  
  \frac{V(z_s) c_q\eye}{l_m}   \t_q \bdot 
  \iint \! \! \f_m^*(\vecrho) e^{-j\vecbeta_{m_q n_q}\bdot\vecrho} \, \d S 
  \nonumber \\
  &= 
  \frac{V(z_s) c_q\eye}{l_m}   \t_q \bdot 
  \ftf_m^*(\vecbeta_{m_q n_q}), \quad
  m = 1, 2, \ldots,  N_J,  
\end{align}
where
\begin{equation}
    \ftf_m(\k) = \iint \!\!
    \f_m(\vecrho) \, e^{j\k \bdot \vecrho} \, \d S 
\end{equation}
is the Fourier transform of the $m$th basis function evaluated at $\k
= \x k_x + \y k_y.$  The Fourier transform is numerically
evaluated using the formulas of \cite{mcsi:91}.



\section{Magnetic Current Mixed Potential Integral Equation}
In this section we consider an FSS that is modeled using equivalent
magnetic surface current.  This choice is convenient when more than
half of the unit cell area is occupied by perfectly conducting metal.

We assume that the primary excitation is an 
incoming Floquet mode with unit amplitude, incident
upon the multilayered structure from either Region~$1$ or $N.$

Using the equivalence principle, the FSS aperture $S$ is filled in with PEC
and an unknown magnetic surface current ($\M$ on the incident side,
and $-\M$ on the far side) is impressed just
outside the metal surface, so as to reproduce the scattered
field exactly in all regions.  The incident field is thus the field
that would exist in the structure with an unperforated ground plane at
$z=z_s$.  It includes reflections from the ground plane on the
incident side of $z_s$ and is zero on the far or transmitted side.

In the case where the incoming wave is incident from Region~$1$, the
boundary value problem to be solved is 
\begin{equation}
  \Idemfactor_z \bdot [\H\inc(x,y,z_s) + \H\ess\{\M\}(x,y,z_s)]
  =   \Idemfactor_z \bdot \H\spw\{-\M\}(x,y,z_s), 
    \quad (\vecrho \in S)
\end{equation}
where 
$H\inc$ is the total ``incident'' magnetic field in the presence of
the unperforated PEC plane at $z=0$,
$\H\ess\{\M\}(\r)$ is the magnetic field evaluated at point $\r \in \{z<z_s\}$ that
would be radiated by the periodic magnetic surface current 
$\M$ impressed on the groundplane at $z=z_s-0$, and 
$\H\spw\{\M\}(\r)$ is the magnetic field evaluated at point $\r \in \{z>z_s\}$ that
would be radiated by the periodic magnetic surface current 
$\M$ impressed on the groundplane at $z=z_s+0.$
In the case where the incoming wave is incident from Region~$N$, the
boundary value problem to be solved is 
\begin{equation}
  \Idemfactor_z \bdot \H\ess\{-\M\}(x,y,z_s)
  =   \Idemfactor_z \bdot [\H\inc(x,y,z_s) + \H\spw\{\M\}(x,y,z_s)], 
    \quad (\vecrho \in S)
\end{equation}

Using the linearity of the magnetic field operator allows
us to rewrite the both integral equations in a unified form:
\begin{equation}
-\Idemfactor_z \bdot [\H\ess\{\M\}(\vecrho) + 
\H\spw\{\M\}(\vecrho)] = 
\Idemfactor_z \bdot \H\inc(\vecrho), \quad (\vecrho \in S)
  \label{eq:MFIE}
\end{equation}
where it is to be understood that $\vecrho$ represents a point in the
plane $z=z_s.$
%continue here
In terms of potentials the boundary value problem becomes
\begin{equation}
  [-j\omega \F(\vecrho) + \gradient\Psi(\vecrho)] \bdot \Idemfactor_z = 
  \H\inc(\vecrho) \bdot \Idemfactor_z, \quad (\vecrho \in S)
  \label{eq:HMPIE}
\end{equation}
where the potentials can be expressed as superposition integrals
 \begin{gather}
   \Idemfactor_z \bdot \F(\vecrho) = 
   \iint \GFxx(\vecrho-\vecrho') \cdot \M(\vecrho') \, \d S',
   \label{eq:Fsup}\\
   \Psi(\vecrho) = \iint \GPsi(\vecrho-\vecrho') \cdot \qm(\vecrho') \, \d S'.
   \label{eq:Psisup}
 \end{gather}
In \eqref{eq:Fsup} and \eqref{eq:Psisup}, $\M$ and $\qm$ are the unknown
induced magnetic surface current and surface charge densities,
respectively, which will be determined
by enforcing
Equation~\eqref{eq:HMPIE} in an approximate manner using the method of
moments. The functions $\GFxx$ and $\GPsi$ are the sum of the
left-looking and right-looking Green's functions as defined in 
Chapter~\ref{chap:gfstratified}:
\begin{subequations}
  \begin{align}
    \GFxx(\vecrho-\vecrho') &=
    \GlFxx(\vecrho-\vecrho',z_s,z_s) +
    \GrFxx(\vecrho-\vecrho',z_s,z_s) \\
    \GPsi(\vecrho-\vecrho') &=
    \GlPsi(\vecrho-\vecrho',z_s,z_s) +
    \GrPsi(\vecrho-\vecrho',z_s,z_s).
  \end{align}
\end{subequations}


 We again employ the triangle subdomain basis functions that
were described in Section~\ref{sec:bf}.

We now expand the unknown magnetic surface current density 
in a series of triangle-pair basis functions:
\begin{equation}
  \label{eq:Mexpand}
  \M(\vecrho) =  \sum_{n=1}^{N_M} \frac{\Vcoef_n}{l_n} \f_n(\vecrho).
\end{equation}
The number of basis functions
$N_M$ is equal to the number of interior edges of the triangulated aperture in the
$z=0$ plane of the unit cell plus the number of edges along the $\xi=0$ and $\eta=0$
borders of the unit cell.  Note that $\Vcoef_n$ has units of \si{V}.


\subsection{Admittance Matrix}
\subsubsection{Reduction to Matrix Equation}
Having expanded the unknown magnetic surface 
current in a series of suitable basis functions, 
it is now desired to determine the coefficients in a way that satisfies 
Equation~\eqref{eq:HMPIE} as closely as possible.  

We choose our testing functions to be the same as our basis functions 
(Galerkin's method).  Then taking the inner product of \eqref{eq:HMPIE}
with each of the functions $\f_m/l_m$  for $m = 1,2,\ldots, N_M$ yields
\begin{equation}
  -j\omega \innerprod{\F,\f_m/l_m} + \innerprod{\gradient\Psi,\f_m/l_m}
  = \innerprod{\H\inc, \f_m/l_m}, \qquad
  m = 1, 2, \ldots,  N_M.   \label{eq:Hmom}
\end{equation}  
We approximate the inner products in the same manner as was done for
the case of electric sources:
\begin{equation}
  \innerprod{\gradient\Psi, \f_m/l_m} \approx 
         e^{-j\theta_m^-} \Psi(\rcm_m) - e^{-j\theta_m^+} \Psi(\rcp_m)  
         \label{eq:Psimom}
\end{equation}
where $\rcpm_m$ is the centroid of triangle $T_m^{\pm}$. Also, we use
\begin{equation}
  \innerprod{\F, \f_m/l_m} \approx \frac{1}{2} 
      \left[ \F(\rcp_m) \bdot \vecrhocp_m e^{-j\theta_m^+} +  
             \F(\rcm_m) \bdot \vecrhocm_m  e^{-j\theta_m^-}
      \right].      \label{eq:Fmom}
\end{equation}  
$\vecrhocpm_m$ is the vector $\vecrho_m^{\pm}$ evaluated at the centroid
of the corresponding triangle.


Substituting the expansion of the current \eqref{eq:Mexpand} into \eqref{eq:Hmom}
yields the desired matrix equation.
\begin{equation}
  \mat{Y} \mat{V} =  \mat{I},  \label{eq:Hmatrix}
\end{equation}
where $\mat{Y} = [\matel{Y}_{mn}]$ is the so-called {\em generalized admittance
matrix}, a square matrix of order $N_M$ with units of \si{S}.  
$\mat{V} = [\Vcoef_m]$ is the column matrix of unknown magnetic current
coefficients with units of \si{V}. 
$\mat{I}= [\matel{I}_m]$ is the excitation vector (or {\em generalized
current vector}) with units of \si{A}.
 The elements of the generalized admittance matrix are
\begin{equation}
  \matel{Y}_{mn} = 
 -j\omega \left(
    \F_{mn}^+ \bdot \frac{\vecrhocp_m}{2} e^{-j\theta_m^+}
    +
    \F_{mn}^- \bdot \frac{\vecrhocm_m}{2} e^{-j\theta_m^-}
  \right)
  + \Psi_{mn}^- e^{-j\theta_m^-} - \Psi_{mn}^+ e^{-j\theta_m^+}
\end{equation}
where
 \begin{align}
   \F_{mn}^{\pm} & \equiv 
   \frac{1}{l_n} \iint  
   \GFxx(\rcpm_m;\r') \, \f_n(\r') \; \d S'  \notag \\
   &= 
     \frac{e^{j\theta_n^+}}{2A_n^+} \iint_{T_n^+}  \!\!\!
     \GFxx(\rcpm_m;\r') \, \vecrho_n^+ \; \d S'
     +
     \frac{e^{j\theta_n^-}}{2A_n^-} \iint_{T_n^-}  \!\!\!
     \GFxx(\rcpm_m;\r')  \, \vecrho_n^- \; \d S'
     \label{eq:Fmnpm}
 \end{align}
and
\begin{align}
  \Psi_{mn}^{\pm} &\equiv \frac{j}{\omega l_n} \iint
  \GPsi(\rcpm_m;\r')
  \gradient'_s \bdot \f_n(\r')  
  \; \d S'
  \notag \\
  &= \frac{j}{\omega}
  \left[
    \frac{e^{j\theta_n^+}}{A_n^+} \iint_{T_n^+}  \!\!\!
    \GPsi(\rcpm_m;\r') \; \d S'
    -
    \frac{e^{j\theta_n^-}}{A_n^-} \iint_{T_n^-}{}  \!\!\!
    \GPsi(\rcpm_m;\r') \; \d S'
  \right]    \label{eq:Psimnpm}
\end{align}

\subsubsection{Efficient Evaluation of the Generalized Admittance Matrix}
\paragraph{Vector Potential Contributions}
Consider first a typical vector integral needed
to evaluate $\matel{Y}_{mn}$.  
The quantity
\begin{equation}
  \F_i^{uv} =   \frac{e^{j\theta_i^\pm}}{2A^v}  \iint_{T^v}  \!\!\!
  \GFxx(\rcu;\r') \, \vecrho_i \; \d S' \label{eq:Fiuv}
\end{equation}
represents the electric vector potential at the centroid $\rcu$ of triangle
$T^u$ due to the $i\text{th}$ basis function defined on triangle $T^v$.
In \eqref{eq:Fiuv} we have temporarily assumed a local numbering scheme
on the source triangle, with vertices at $\r_1$, $\r_2$, and $\r_3$,
and where $\vecrho_i$ is defined in Equation~\eqref{eq:vecrhosubi}.
Using the expressions for the electric vector potential Green's
function given in Chapter~\ref{chap:gfstratified},  Equation~\eqref{eq:Fiuv} can 
be written as 
the sum of a frequency dependent term arising from the modal
difference series, added to several contributions arising from the
spatial series:
\begin{equation}
  \F_i^{uv} = -\frac{\bar{\epsilon}}{\pi} e^{j\theta^{\pm}_i}
  \Biggl[
    \frac{\pi\epsilon_0}{\bar{\epsilon}}  \I_{1i}^{'uv} + 
    \frac{1}{2A^v} \! \iint_{T^v}  \!
    \frac{\vecrho_i}{\rhocu_{00}} \, \d S' +
       u \J_i^{uv} +
    \frac{c_3\ess \epsilon_s + c_3\spw \epsilon_{s+1}}{2\bar{\epsilon}u}
     \K_i^{uv}
  \Biggr] \label{eq:Fiuv2}
\end{equation}
where
\begin{equation}
     \I_{1i}^{'uv} =
     \frac{1}{2A^v}  \iint_{T^v}  \!\!\!
     \Sigma'_{M1}(\rcu;\r')
     \, \vecrho_i \; \d S' 
\end{equation}
and $\Sigma'_{M1}$ is defined in Equation~\eqref{eq:finalmodalmagnetic}.
Note that the
only frequency dependence in \eqref{eq:Fiuv2} occurs in the terms
$\I_{1i}^{uv}$ and $c_3$.  Therefore, the contrubutions due to the
singular integral, $\J_i^{uv}$ and $\K_i^{uv}$ can be computed once,
stored, and combined appropriately at each new analysis frequency.
Although the value of $\I_{1i}^{'uv}$ must be computed anew 
at each frequency, the modal
difference series occurring in its integrand can be very rapidly evaluated by
simple interpolation into a two-dimensional table obtained from the
FFT (Section~\ref{sec:modalFFT}).


\paragraph{Scalar Potential Contributions}
In addition to the vector integrals just discussed, it is also 
necessary to calculate scalar integrals of the form
\begin{equation}
  \Psi_i^{uv} = \pm \frac{je^{j\theta^{\pm}_i}}{\omega A^v} \!
  \iint_{T^v} \!\!
  \GPsi(\r^{cu};\r') \; \d S'.  \label{eq:Psiiuv}
\end{equation}
$\Psi_i^{uv}$ is the 
magnetic scalar potential evaluated at the centroid of
triangle $T^u$ due to the surface magnetic charge density associated with
the $i$th basis function defined on triangle $T^v$.  As before, the
sign is taken to be positive if the basis function in question defines
positive reference current to flow {\em out} of $T^v$ through the 
$i$th edge. 
We proceed as before, using the results of Chapter~\ref{chap:gfstratified} for the
magnetic scalar potential Green's function to write the above integral
as 
\begin{equation}
  \Psi_i^{uv} = \frac{\pm 2 j e^{j\theta^{\pm}_i}}{\pi\omega\tilde{\mu}} 
  \Biggl[
  \frac{\pi\tilde{\mu}}{\mu_0}  
  I_{2}^{'uv} + 
  \frac{1}{2A^v}  \! \iint_{T^v}  \!
      \frac{\d S'}{\rhocu_{00}} +
      u J_i^{uv} +
      \frac{\tilde{\mu}}{2u} 
      \left(
        \frac{d_3\ess}{\mu_s} + \frac{d_3\spw}{\mu_{s+1}}
      \right)
      K_i^{uv}
    \Biggr] \label{eq:Psiiuv2}
 \end{equation}
where
\begin{equation}
     I_{2}^{'uv} =
     \frac{1}{2A^v}  \iint_{T^v}  \!\!\!
     \Sigma'_{M2}(\rcu;\r')
     \; \d S' 
\end{equation}
and $\Sigma'_{M2}$ is defined in Equation~\eqref{eq:finalmodalmagnetic}.

All of the integrals in \eqref{eq:Psiiuv2} have been previously
discussed.  Note that the only frequency dependence (apart from the leading factor of
$1/\omega$) occurs in the terms $I_{2}^{'uv}$ and $d_3$. 

  
\subsection{\label{subsec:gcm}Generalized Current Vector}
The elements of the generalized current vector are defined as 
\begin{equation}
  \matel{I}_m =  \innerprod{\H\inc, \f_m/l_m} = \frac{1}{l_m} 
  \iint \H\inc \bdot \f_m^* \, \d S, \quad
  m = 1, 2, \ldots,  N_J.  
  \label{eq:Im}
\end{equation}
The incident field $\H\inc$ is the field that would exist in the
presence of the unperforated FSS metalization due to an incoming, normalized
Floquet mode from Region~$1$ or~$N$ with unit amplitude coefficient.
This scenario was treated in Chapter~\ref{chap:incgsm} and we use the results
derived there.  Let $i \in \{1,N\}$ be the region index designating
the source of the incoming Floquet mode.  The mode index is $q =
(p_q,m_q,n_q)$ where $p_q=1$ for TE modes and $p_q=2$ for TM modes.  The
transverse portion of the incident magnetic field evaluated in the
$z=z_s$ plane is:
\begin{equation}
  \Idemfactor_z \bdot \H\inc(\vecrho) = I(z_s) \h_q\eye(\vecrho)
  = I(z_s) Y_q\eye c_q\eye \z\cross\t_q e^{-j\vecbeta_{m_q n_q} \bdot
  \vecrho} \label{eq:Hincmom}
\end{equation}
where the modal polarization vector $\t_q$ was defined in \eqref{eq:tdef}
and $I(z_s)$ is the equivalent transmission line current evaluated at
$z=z_s$ using the formulas presented in Chapter~\ref{chap:incgsm}.

Upon substituting the incident field of \eqref{eq:Hincmom} into
\eqref{eq:Im} we find that
\begin{align}
  \matel{I}_m &=  
  \frac{I(z_s) Y_q\eye c_q\eye}{l_m} \z \cross \t_q \bdot 
  \iint \! \! \f_m^*(\vecrho) e^{-j\vecbeta_{m_q n_q}\bdot\vecrho} \, \d S 
  \nonumber \\
  &= 
  \frac{I(z_s) Y_q\eye c_q\eye}{l_m} \z \cross  \t_q \bdot 
  \ftf_m^*(\vecbeta_{m_q n_q}), \quad
  m = 1, 2, \ldots,  N_M,  
\end{align}
where
\begin{equation}
    \ftf_m(\k) = \iint \!\!
    \f_m(\vecrho) \, e^{j\k \bdot \vecrho} \, \d S 
\end{equation}
is the Fourier transform of the $m$th basis function evaluated at $\k
= \x k_x + \y k_y.$  The Fourier transform is numerically
evaluated using the formulas of \cite{mcsi:91}.
