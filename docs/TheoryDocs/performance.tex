\documentclass[11pt]{article}
\usepackage{sslmemo}
\usepackage{units}
\usepackage{amsmath,amsfonts} % For DeclareMathOperator and blackboard
                              % bold
\usepackage[LY1]{fontenc}               % tell TeX about LY1 encoding
%\usepackage[mtbold,mtplusscr,subscriptcorrection,LY1]{mathtime}
\usepackage[heavybold,mtpluscal,subscriptcorrection,LY1]{mathtime}

\usepackage{url}
\usepackage[small,bf]{caption}
\usepackage[dvips]{graphicx}
\usepackage{pstricks}
\usepackage[dvips,letterpaper]{hyperref}

\organization{Antenna RF Design}
\subject{Calculation of performance parameters for PSS}
\addressee{File}
\from{P. Simon\rlap{\kern20pt\raisebox{-8pt}[0pt][0pt]{\includegraphics{petersimonsig.eps}}}}
\distribution{%
  L. Ersoy \\
  B. Lee \\
  P. Nanavati \\
  E. Nygren \\
  M. Suarez-Barnes
  }
\date{June 18, 2001\rlap{\;\;(rev 1/15/2003)}}
\filename{performance.tex}
\memonumber{AAS/01/0020/IRAD}

%%%%%%%%%%%%%%%%%%%%%%%%%%%%%%%%%%%%%%
%% Revision History
%% 1/15/2003: Corrected several typos on page 3.
%%%%%%%%%%%%%%%%%%%%%%%%%%%%%%%%%%%%%%

\allowdisplaybreaks
\hyphenation{mode}

%%%%%%%%%%%%%%%%%%%%%% Begin macro definitions  %%%%%%%%%%%%%%%%%%%%%%%%%%%%%%%
%\usepackage{bm} % Correct bold math
\newcommand{\pssfss}{PSSFSS}
\renewcommand{\d}{\text{d}}  % for use in integrals
\newcommand{\Realnum}{\mathbb{R}}
\newcommand{\Complexnum}{\mathbb{C}}
\newcommand{\Integers}{\mathbb{Z}}
\newcommand{\Fourier}{\mathcal{F}}
\DeclareMathOperator{\RealOp}{Re}
\DeclareMathOperator{\sgn}{sgn}
\newcommand{\Real}[1]{\RealOp\left\{#1\right\}}
\DeclareMathOperator{\ImagOp}{Im}
\newcommand{\Imag}[1]{\ImagOp\left\{#1\right\}}
\newcommand{\diag}[1]{\text{diag}\left\{#1\right\}}
\newcommand{\0}{\boldsymbol{0}}
\newcommand{\A}{\boldsymbol{A}}
\newcommand{\E}{\boldsymbol{E}}
\newcommand{\e}{\boldsymbol{e}}
\renewcommand{\k}{\boldsymbol{k}}
\newcommand{\f}{\boldsymbol{f}}  % Basis function.
\newcommand{\ftf}{\boldsymbol{\tilde{f}}}  % Basis function Fourier transform.
\newcommand{\F}{\boldsymbol{F}}
\renewcommand{\H}{\boldsymbol{H}}
\newcommand{\h}{\boldsymbol{h}}
\newcommand{\hhat}{\rlap{$\boldsymbol{\hat{h}}$}\phantom{\boldsymbol{h}}}
\newcommand{\vhat}{\rlap{$\boldsymbol{\hat{v}}$}\phantom{\boldsymbol{v}}}
\newcommand{\J}{\boldsymbol{J}}
\newcommand{\Jt}{\boldsymbol{\tilde{J}}}
\newcommand{\I}{\boldsymbol{I}}
\newcommand{\K}{\boldsymbol{K}}
\newcommand{\Js}{\J_{\text{s}}}
\newcommand{\ftJs}{\boldsymbol{\tilde{J}}_{\text{s}}}
\renewcommand{\L}{\boldsymbol{L}}
\newcommand{\M}{\boldsymbol{M}}
\newcommand{\Ms}{\boldsymbol{M}_{\text{s}}}
\newcommand{\ftMs}{\boldsymbol{\tilde{M}}_{\text{s}}}
\renewcommand{\r}{\boldsymbol{r}}
\newcommand{\x}{\boldsymbol{\hat{x}}}
\newcommand{\y}{\boldsymbol{\hat{y}}}
\newcommand{\z}{\boldsymbol{\hat{z}}}
\newcommand{\abs}[1]{\lvert#1\rvert}
\newcommand{\norm}[1]{\left\lVert#1\right\rVert}
\newcommand{\gradient}{\boldsymbol{\nabla}}
\newcommand{\cross}{\boldsymbol{\times}}
\newcommand{\bdot}{\boldsymbol{\cdot}}
\newcommand{\curl}{\boldsymbol{\nabla} \cross}
\newcommand{\divergence}{\boldsymbol{\nabla} \bdot}
\newcommand{\laplace}{\nabla^2}  % Scalar Laplacian Operator
\newcommand{\vlaplace}{\gradient^2}  % Vector Laplacian Operator
\newcommand{\qe}{q_{\text{e}}}  % Electric charge density
\newcommand{\qm}{q_{\text{m}}}  % Magnetic charge density
\newcommand{\union}{\bigcup}  % Magnetic charge density
\newcommand{\s}{\boldsymbol{s}}  % Direct lattice vector
\newcommand{\vecbeta}{\boldsymbol{\beta}}  % Reciprocal lattice vector
\newcommand{\betahat}{\boldsymbol{\hat{\beta}}}  % Reciprocal lattice unit vector
\newcommand{\vecrho}{\boldsymbol{\rho}}  % Reciprocal lattice vector
\newcommand{\colvec}[1]{\begin{bmatrix} #1 \end{bmatrix}}
%\newcommand{\colvec}[1]{\left[\begin{array}{c} #1 \end{array}\right]}
\newcommand{\TE}{^{\text{TE}}}
\newcommand{\TM}{^{\text{TM}}}
\newcommand{\mat}[1]{\boldsymbol{\mathcal{#1}}}
\newcommand{\matel}[1]{\mathcal{#1}}
\newcommand{\Icoef}{\matel{I}}  % Electric current expansion coefficient.
\newcommand{\Vcoef}{\matel{V}}  % Magnetic current expansion coefficient.
\newcommand{\one}{^{(1)}} % Region 1 designator
\newcommand{\two}{^{(2)}} % Region 2 designator
\newcommand{\eye}{^{(i)}} % Region i designator
\newcommand{\epw}{^{(i+1)}} % Region i+1 designator
\newcommand{\enn}{^{(N)}} % Region N designator
\newcommand{\ess}{^{(s)}} % Region s designator
\newcommand{\spw}{^{(s+1)}} % Region s+1 designator
\newcommand{\tmi}{^{(3-i)}} % Region 3-i designator
\newcommand{\three}{^{(3)}} % Region 3 designator
\newcommand{\four}{^{(4)}} % Region 4 designator
\newcommand{\transpose}{^{\text{\sffamily{T}}}}
\newcommand{\hermitian}{^{\text{\sffamily{H}}}}
\newcommand{\inc}{^{\text{inc}}}
\newcommand{\refl}{^{\text{r}}}
\newcommand{\trans}{^{\text{t}}}
\newcommand{\scat}{^{\text{sc}}}
\renewcommand{\t}{\boldsymbol{\hat{t}}}
\newcommand{\n}{\boldsymbol{\hat{n}}}
\newcommand{\pdx}[1]{\frac{\partial #1}{\partial x}}
\newcommand{\pdz}[1]{\frac{\partial #1}{\partial z}}
\newcommand{\pdu}[1]{\frac{\partial #1}{\partial u}}
\newcommand{\summn}{\sum_{m,n}}
\newcommand{\dyad}[1]{\boldsymbol{\bar{#1}}}
\newcommand{\Idemfactor}{\dyad{I}}
\newcommand{\GAxx}{G^A_{xx}}
\newcommand{\GtAxx}{\tilde{G}^{A}_{xx}}
\newcommand{\GFxx}{G^F_{xx}}
\newcommand{\GrFxx}{\raisebox{0pt}[0pt][0pt]{$\vecr{G}$}\vphantom{G}^F_{xx}}
\newcommand{\GlFxx}{\raisebox{0pt}[0pt][0pt]{$\vecl{G}$}\vphantom{G}^F_{xx}}
\newcommand{\GPhi}{{G}^{\Phi}}
\newcommand{\GtPhi}{{G}^{\Phi}}
\newcommand{\GPsi}{{G}^{\Psi}}
\newcommand{\GrPsi}{\raisebox{0pt}[0pt][0pt]{$\vecr{G}$}\vphantom{G}^\Psi}
\newcommand{\GlPsi}{\raisebox{0pt}[0pt][0pt]{$\vecl{G}$}\vphantom{G}^\Psi}
\newcommand{\rcp}{\r^{\text{c}+}} % Positive centroid r vector.
\newcommand{\rcm}{\r^{\text{c}-}} % Negative centroid r vector.
\newcommand{\rcpm}{\r^{\text{c}\pm}} % Plus or Negative centroid r vector.
\newcommand{\vecrhocp}{\vecrho^{\text{c}+}} % Positive centroid rho vector.
\newcommand{\vecrhocm}{\vecrho^{\text{c}-}} % Positive centroid rho vector.
\newcommand{\vecrhocpm}{\vecrho^{\text{c}\pm}} % Positive centroid rho vector.
\newcommand{\innerprod}[1]{\left< #1 \right>}
%\renewcommand{\iint}{\int \!\!\! \int}
\newcommand{\rcu}{\r^{\text{c}u}} % Centroid r vector for triangle ``u''.
\newcommand{\rhocu}{\rho^{\text{c}u}} % Centroid rho for triangle ``u''.
\renewcommand{\P}{\boldsymbol{P}}
\renewcommand{\l}{\boldsymbol{l}}
\newcommand{\lhat}{\boldsymbol{\hat{l}}}
\renewcommand{\u}{\boldsymbol{\hat{u}}}  % unit vector.
\newcommand{\is}{i_{\text{s}}}  % Source region designator
\renewcommand{\it}{i_{\text{t}}}  % Transmitted region designator
\newcommand{\eyes}{^{(\is)}} % Source Region designator
\newcommand{\eyet}{^{(\it)}} % Transmitted Region designator
% Vector accent pointing to left:
\newcommand{\vecl}[1]{\overset{\raisebox{-0.6ex}{\tiny$\boldsymbol{\leftarrow}$}}{#1}}  
% Vector accent pointing to right:
\newcommand{\vecr}[1]{\overset{\raisebox{-0.6ex}{\tiny$\mkern5mu\boldsymbol{\rightarrow}$}}{#1}}  
\newcommand{\Zright}{\vecr{Z}}
\newcommand{\Zleft}{\vecl{Z}}
%\newcommand{\Yleft}{\vecl{Y}}
%\newcommand{\Yright}{\vecr{Y}}
 \newcommand{\Yleft}%
     {\raisebox{0pt}[0pt][0pt]{\rlap{$\vecl{\phantom{Y}}$}}Y}  % TLGF
 \newcommand{\Yright}%
 {\raisebox{0pt}[0pt][0pt]{\rlap{$\vecr{\phantom{Y}}$}}Y}  % TLGF
\newcommand{\Vi}{V_{\text{i}}}  % TLGF
\newcommand{\Vv}{V_{\text{v}}}  % TLGF
\newcommand{\Ii}{I_{\text{i}}}  % TLGF
\newcommand{\Iv}{I_{\text{v}}}  % TLGF

 \newcommand{\Ilv}%
 {\raisebox{0pt}[0pt][0pt]{\rlap{$\vecl{I}$}}\phantom{I}_{\text{v}}}  % TLGF
 \newcommand{\Irv}%
 {\raisebox{0pt}[0pt][0pt]{\rlap{$\vecr{I}$}}\phantom{I}_{\text{v}}}  % TLGF
\newcommand{\tauhat}%
 {\rlap{$\boldsymbol{\hat{\tau}}$}\phantom{\boldsymbol{\tau}}} % Pol. vector
\newcommand{\uhat}{\boldsymbol{\hat{u}}}  % Polarization vector
\newcommand{\what}{\boldsymbol{\hat{w}}}  % Polarization vector
\newcommand{\khat}{\boldsymbol{\hat{k}}}  % Polarization vector
\newcommand{\sigmahat}%
 {\rlap{$\boldsymbol{\hat{\sigma}}$}\phantom{\boldsymbol{\sigma}}} % Pol. vec.


%% \tlsection{x1}{x4}{n} draws a transmission line section
%% from x1 to x4, with nodes at x1 and x4, and labels the 
%% section as number ``n''
\newcommand{\tlsection}[3]{%
                                % Draw top line:
  \mmultido{\rrone=#1+0.3,\rrtwo=#2+-0.3}{1}{%
    \qdisk(#1,0.8){1.5pt} \psline(#1,.8)(\rrone,.8) %
    \psframe(\rrone,0.7)(\rrtwo,0.9) %
    \psline(\rrtwo,0.8)(#2,0.8) \qdisk(#2,0.8){1.5pt} %
                                % Draw bottom line
    \qdisk(#1,-0.8){1.5pt} \psline(#1,-0.8)(\rrone,-0.8) %
    \psframe(\rrone,-0.9)(\rrtwo,-0.7) %
    \psline(\rrtwo,-0.8)(#2,-0.8) \qdisk(#2,-0.8){1.5pt} %
                                % Labels
    \pcline[linestyle=none](\rrone,0)(\rrtwo,0)
    \lput{:U}{\cstack{$Z_{pmn}^{(#3)}$ \\[0.5ex] $\gamma_{mn}^{(#3)}$}}
                                % Dimensions
    \pcline[linewidth=0.5pt]{<->}(\rrone,1.3)(\rrtwo,1.3)
    \pcline{|-|}(\rrone,1.3)(\rrtwo,1.3)
    \lput*{:U}{$h^{(#3)}$}
    }
  }

%% \Ytlsection{x1}{x4}{n} draws a transmission line section
%% from x1 to x4, with nodes at x1 and x4, and labels the 
%% section as number ``n'' using a Y rather than a Z
\newcommand{\Ytlsection}[3]{%
                                % Draw top line:
  \mmultido{\rrone=#1+0.3,\rrtwo=#2+-0.3}{1}{%
    \qdisk(#1,0.8){1.5pt} \psline(#1,.8)(\rrone,.8) %
    \psframe(\rrone,0.7)(\rrtwo,0.9) %
    \psline(\rrtwo,0.8)(#2,0.8) \qdisk(#2,0.8){1.5pt} %
                                % Draw bottom line
    \qdisk(#1,-0.8){1.5pt} \psline(#1,-0.8)(\rrone,-0.8) %
    \psframe(\rrone,-0.9)(\rrtwo,-0.7) %
    \psline(\rrtwo,-0.8)(#2,-0.8) \qdisk(#2,-0.8){1.5pt} %
                                % Labels
    \pcline[linestyle=none](\rrone,0)(\rrtwo,0)
    \lput{:U}{\cstack{$Y_{q}^{(#3)}$ \\[0.5ex] $\gamma_{mn}^{(#3)}$}}
                                % Dimensions
    \pcline[linewidth=0.5pt]{<->}(\rrone,1.3)(\rrtwo,1.3)
    \pcline{|-|}(\rrone,1.3)(\rrtwo,1.3)
    \lput*{:U}{$h^{(#3)}$}
    }
  }


%% \vsource{x} places a voltage generator at location x
\newcommand{\vsource}[1]{%
  \psline(#1,-0.8)(#1,-0.2) \qdisk(#1,-0.8){1.5pt}
  \psline(#1,0.8)(#1,0.2) \qdisk(#1,0.8){1.5pt}
  \pscircle[fillcolor=white](#1,0){0.2}
  %\psline{->}(#1,-0.13)(#1,0.13)
   \rput(#1,0){\renewcommand{\arraystretch}{0}%
     \raisebox{-3pt}[0pt][0pt]{%
     \begin{tabular}{@{}c@{}}
       $\scriptstyle+$ \\ \raisebox{1pt}[0.8\height][0pt]{$\scriptstyle-$}
     \end{tabular}}%
     }
  }



%% \invertedvsource{x} places an upside down voltage generator at location x
\newcommand{\invertedvsource}[1]{%
  \psline(#1,-0.8)(#1,-0.2) \qdisk(#1,-0.8){1.5pt}
  \psline(#1,0.8)(#1,0.2) \qdisk(#1,0.8){1.5pt}
  \pscircle[fillcolor=white](#1,0){0.2}
  %\psline{->}(#1,-0.13)(#1,0.13)
   \rput(#1,0){\renewcommand{\arraystretch}{0}%
     \raisebox{-2pt}[0pt][0pt]{%
     \begin{tabular}{@{}c@{}}
       $\scriptstyle-$ \\ \raisebox{1pt}[0.8\height][0pt]{$\scriptstyle+$}
     \end{tabular}}%
     }
  }



%% \isource{x} places a current generator at location x
\newcommand{\isource}[1]{%
  \psline(#1,-0.8)(#1,-0.2) \qdisk(#1,-0.8){1.5pt}
  \psline(#1,0.8)(#1,0.2) \qdisk(#1,0.8){1.5pt}
  \pscircle[fillcolor=white](#1,0){0.2}
  \psline{->}(#1,-0.13)(#1,0.13)
  }

%% \shuntload{x}{label} places a load at location x with label ``label''
\newcommand{\shuntload}[2]{%
  \psline(#1,-0.8)(#1,0.8) \qdisk(#1,-0.8){1.5pt} \qdisk(#1,0.8){1.5pt}
  \rput*(#1,0){\psframebox[framesep=2pt,boxsep=false]{#2}}
  }
%% \zlabel{x}{label} places label at location x:
\newcommand{\zlabel}[2]{%
  \psline[linestyle=dashed,linewidth=0.6pt](#1,-1.6)(#1,-1)
  \rput[t](#1,-1.7){#2}
  }

\newcommand{\cstack}[1]{\begin{tabular}{@{}c@{}} #1 \end{tabular}}
\newcommand{\khatinc}{\boldsymbol{\hat{k}}^{\text{i}}}
\newcommand{\thetainc}{\theta^{\text{i}}}

%% \seriesload{x1}{x2}{x3}{label} draws a series load 
%% from x1 to x3, centered at x2, 
%% with nodes at x1 and x3, and labels the 
%% load using the third argument:
\newcommand{\seriesload}[4]{%
                                % Draw top line:
  \qdisk(#1,0.8){1.5pt} \psline(#1,.8)(#3,.8) \qdisk(#3,0.8){1.5pt} %
                                % Draw bottom line
  \qdisk(#1,-0.8){1.5pt} \psline(#1,-0.8)(#3,-0.8) \qdisk(#3,-0.8){1.5pt} %
                                % Labels
  \rput*(#2,0.8){\psframebox[boxsep=false]{#4}} %
  }



 %%%%%%%%%%%%%%%%%%%%%% End macro definitions  %%%%%%%%%%%%%%%%%%%%%%%%%%%%%%%

\renewcommand{\inc}{^{\text{i}}}
\renewcommand{\refl}{^{\text{r}}}
\renewcommand{\abs}[1]{\bigl\lvert#1\bigr\rvert}
\newcommand{\piotwo}{{\textstyle\frac{\pi}{2}}}


\begin{document}
\section{Introduction}
A set of previous notes has detailed the analytical formulation for
calculating the generalized scattering matrix (GSM) of a frequency
selective surface (FSS). 
Although knowledge of the magnitude and phase
of the GSM elements provides a complete description of the FSS, it is
often more convenient for the user to examine other performance
parameters. This memorandum defines a number of user-oriented
performance parameters and describes how they are calculated from
elements of the GSM.
The latest version of this document can be found at the following URL: 
\url{http://mogh/~simonp/spaf}.

We assume henceforth that there are exactly two propagating modes in
layers~1 and $N$: the principal ($m=n=0$) TE and TM modes.  We also
assume that these two regions are lossless, so that their wavenumbers 
$k_1$ and $k_N$ and intrinsic impedances $\eta_1$ and $\eta_N$ are
all positive numbers. The
scattering relation for the incident and reflected components of the
dominant TE/TM modes is determined from the GSM:
  \begin{align}
    \colvec{b\one_{100} \\ b\one_{200} \\ b\enn_{100} \\ b\enn_{200}} 
    %
    &= 
    %
    \begin{bmatrix}
      \mat{S}^{11} & \mat{S}^{12}    \\
      \mat{S}^{21} & \mat{S}^{22}    
    \end{bmatrix}
    %
    \colvec{a\one_{100} \\ a\one_{200} \\ a\enn_{100} \\ a\enn_{200}} 
    %
    %
    =
    %
    \begin{bmatrix}
      \matel{S}^{11}_{11} & \matel{S}^{11}_{12} & \matel{S}^{12}_{11}
      & \matel{S}^{12}_{12} \\ 
      \matel{S}^{11}_{21} & \matel{S}^{11}_{22} & \matel{S}^{12}_{21}
      & \matel{S}^{12}_{22} \\ 
      \matel{S}^{21}_{11} & \matel{S}^{21}_{12} & \matel{S}^{22}_{11}
      & \matel{S}^{22}_{12} \\ 
      \matel{S}^{21}_{21} & \matel{S}^{21}_{22} & \matel{S}^{22}_{21}
      & \matel{S}^{22}_{22} 
    \end{bmatrix}
    \colvec{a\one_{100} \\ a\one_{200} \\ a\enn_{100} \\ a\enn_{200}}.
    \label{eq:fssscat}
  \end{align}
The notation and definitions of \cite{simo:98}, \cite{simo:00}, and \cite{simo:00a} are
adopted for use throughout these notes.

\section{Definition of Horizontal and Vertical Polarization}  The GSM
relates the incident and scattered fields using a TE/TM plane wave
decomposition. It is sometimes more convenient (say, when analyzing a
meanderline polarizer in other than the principal planes) to consider
scattering and reflection using a basis consisting of vertical and
horizontal polarization. 

We need to define four sets of horizontal and vertical unit vectors.
These are $(\hhat\inc_1,\vhat\inc_1),$ $(\hhat\refl_1,\vhat\refl_1),$
$(\hhat\inc_N,\vhat\inc_N),$ and $(\hhat\refl_N,\vhat\refl_N),$ which
correspond respectively to Region~$1$ incident and reflected waves and
Region~$N$ incident and reflected waves.  These vectors should be
defined so that the following conditions hold:
\begin{enumerate}
\item The horizontal and vertical unit vectors are perdendicular to
  each other and to their associated wave vector. \label{firstitem}
\item The horizontal unit vectors revert to $\x$ for wave vectors
  lying in the principal $y$--$z$ plane. 
\item The vertical unit vectors revert to $\y$ for wave vectors lying
  in the $x$--$z$ plane.
\item The $x$ and $y$ components of the Region~$1$ horizontal unit
  vectors for incident and reflected waves should be equal.
\item The $x$ and $y$ components of the Region~$N$ horizontal unit
  vectors for incident and reflected waves should be equal.
\item The $x$ and $y$ components of the Region~$1$ vertical unit
  vectors for incident and reflected waves should be equal.
\item The $x$ and $y$ components of the Region~$N$ vertical unit
  vectors for incident and reflected waves should be equal.
\item Whenever the wave numbers in the input and output regions are
  identical ($k_1 = k_N$), then the corresponding $\hhat$ vectors
  should also be (as should the corresponding $\vhat$ vectors).
  \label{lastitem}
\end{enumerate}
These conditions can be met by employing the so-called ``Ludwig~3''
unit vectors.
We define the horizontal and vertical unit vectors $\hhat$ and $\vhat$ using
the Ludwig~3 definition \cite{ludw:73} as
\begin{subequations}
  \begin{align}
    \hhat(\theta,\phi) &= 
    \x [1- \cos^2\phi (1 - \cos\theta)] 
    - \y (1-\cos\theta)\sin\phi\cos\phi 
    - \z \sin\theta \cos\phi \\
    \vhat(\theta,\phi) &= 
    - \x (1-\cos\theta)\sin\phi\cos\phi 
    + \y [1- \sin^2\phi (1 - \cos\theta)] 
    - \z \sin\theta \sin\phi.
  \end{align}
\end{subequations}
\begin{figure}[tbph]
  \begin{center} 
    \leavevmode 
    \pspicture(-4.2,-2.5)(4.2,2)
    %
    \rput(0,0){FSS}  % FSS title
                                % Region 1 labels
    \rput[r](-2.8,0){
      \begin{tabular}{@{}c@{}}
        Region~$1$ \\
        $k_1$, $\eta_1$
      \end{tabular}}
                                % Interface 1
    \psline[linewidth=1pt](-1,-2)(-1,2)  \uput[d](-1,-2){$z=z_1$}
                                % Begin Interface N-1:
    \psline[linewidth=1pt](1,-2)(1,2)    \uput[d](1,-2){$z=z_{N-1}$}    
                                % Region 1 labels
    \rput[l](2.5,0){
      \begin{tabular}{@{}c@{}}
        Region~$N$ \\
        $k_N$, $\eta_N$
      \end{tabular}}
                                % Positive z axis:
    \psline{->}(1,0)(2,0)  \rput[l](2,0){$\;z$}
                                % Negative z axis:
    \psline{->}(-1,0)(-2,0)  \rput[r](-2,0){$-z\;$}
                                % Region N incident wave vector
    \psline[linewidth=2pt]{->}(2.5,-1.5)(1,0) 
    \rput[l](2.5,-1.5){$\;\k_N\inc$}
                                % Region N reflected wave vector
    \psline[linewidth=2pt]{<-}(2.5,1.5)(1,0)
    \rput[l](2.5,1.5){$\;\k_N\refl$}
                                % Region 1 incident wave vector
    \psline[linewidth=2pt]{->}(-2,-1.5)(-1,0)
    \rput[r](-2,-1.5){$\k_1\inc\;$}
                                % Region 1 reflected wave vector
    \psline[linewidth=2pt]{<-}(-2,1.5)(-1,0)
    \rput[r](-2,1.5){$\k_1\refl\;$}
                                % Theta_1^inc arc and label:
    \psarc[linewidth=0.6pt]{<->}(-1,0){.6}{181}{233}
    \rput[r](-1.6,-0.4){$\theta_1\inc$}
                                % Theta_1^ref arc and label:
    \psarc[linewidth=0.6pt]{<->}(-1,0){.6}{127}{178}
    \rput[r](-1.6,.4){$\theta_1\refl$}
                                % Theta_N^inc arc and label:
    \psarc[linewidth=0.6pt]{<->}(1,0){.6}{1}{42}
    \rput[l](1.6,-0.4){$\theta_N\inc$}
                                % Theta_N^ref arc and label:
    \psarc[linewidth=0.6pt]{<->}(1,0){.6}{-42}{-1}
    \rput[l](1.6,.4){$\theta_N\refl$}
    \endpspicture
    \caption{Incident and reflected wave vectors in Regions~$1$
    and~$N$.}
  \label{fig:wavevec}
  \end{center}
\end{figure}
Note that the direction of the horizontal and vertical unit vectors
depends on the ``look angles'' $\theta$ and $\phi$.  We now define
appropriate look angles for Regions~1 and~$N$ for both the incident
and reflected waves.  We begin by defining incident and reflected wave
vectors in both regions as shown in Figure~\ref{fig:wavevec}.
\begin{subequations}
\begin{gather}
  \k\inc_1 = \vecbeta_{00} - \z j \gamma\one_{00}, \quad 
  \k\refl_1 = \vecbeta_{00} + \z j \gamma\one_{00} \\
  \k\inc_N = \vecbeta_{00} -\z j \gamma\enn_{00}, \quad
  \k\refl_N = \vecbeta_{00} + \z j \gamma\enn_{00} 
\end{gather}
\end{subequations}
The look angles are now defined via
\begin{subequations}
  \begin{alignat}{2}
    \cos\theta_1\inc &= \frac{\z \bdot \k_1\inc}{k_1} 
    = \frac{-j\gamma\one_{00}}{k_1}, & \qquad
    \sin\theta_1\inc &= \frac{\beta_{00}}{k_1} \\
    \cos\phi_1\inc &= \x \bdot \frac{\vecbeta_{00}}{\beta_{00}}
    = \frac{\beta_{00x}}{\beta_{00}}, & 
    \sin\phi_1\inc &= \y \bdot \frac{\vecbeta_{00}}{\beta_{00}} 
    = \frac{\beta_{00y}}{\beta_{00}} \\
    \theta_1\refl &= \theta_1\inc, & \qquad
    \phi_1\refl &= \phi_1\inc + \pi \\
    \cos\theta_N\refl &= \frac{\z \bdot \k_N\refl}{k_N} 
    = \frac{-j\gamma\enn_{00}}{k_N}, & \qquad
    \sin\theta_N\refl &= \frac{\beta_{00}}{k_N} \\
    \phi_N\refl &= \phi_1\inc, &  & \\
    \theta_N\inc &= \theta_N\refl, & \qquad
    \phi_N\inc &= \phi_N\refl + \pi  = \phi_1\refl
  \end{alignat} 
\end{subequations}
It is straightforward to show that
criteria~\ref{firstitem}--\ref{lastitem} listed above are satisfied by
choosing
\begin{gather}
  (\hhat\inc_1,\vhat\inc_1) = 
  (\hhat(\theta\inc_1,\phi\inc_1),\vhat(\theta\inc_1,\phi\inc_1)), \quad
  (\hhat\refl_1,\vhat\refl_1) = 
  (\hhat(\theta\refl_1,\phi\refl_1),\vhat(\theta\refl_1,\phi\refl_1)), \\
  (\hhat\inc_N,\vhat\inc_N) = 
  (\hhat(\theta\inc_N,\phi\inc_N),\vhat(\theta\inc_N,\phi\inc_N)), \quad
  (\hhat\refl_N,\vhat\refl_N) = 
  (\hhat(\theta\refl_N,\phi\refl_N),\vhat(\theta\refl_N,\phi\refl_N)).
\end{gather}


\section{Arbitrary polarization}
In order to treat linearly polarized electric fields with an arbitrary
tilt angle, we define the $\tauhat$ direction as 
\begin{equation}
  \tauhat(\tau) = \hhat \cos \tau + \vhat \sin \tau
\end{equation}
where $\tau$ is the angle of $\tauhat$ with respect to the horizontal
direction. 

\subsection{Region~1 Incidence}

Consider a plane wave incident upon the FSS from Region~$1$:
\begin{equation}
  \E\inc = \tauhat_1\inc e^{-j\k_1\inc \bdot (\r - \z z_1)}, \quad z<z_1
  \label{eq:Einc1}
\end{equation}
where
\begin{equation}
  \tauhat_1\inc = \hhat_1\inc \cos \tau_1\inc + \vhat_1\inc \sin \tau_1\inc.
\end{equation}
In order to find the fields scattered from the FSS by the incident
field in \eqref{eq:Einc1}, we express the incident field as the sum of
a TE and TM wave:
\begin{equation}
  (\x\x+\y\y) \bdot \E\inc = a_1\one \e\one_{100} + a_2\one \e\one_{200}
  \label{eq:Einc1expand}
\end{equation}
where it will be recalled that a subscript of $(100)$ denotes the
principal TE mode and $(200)$ denotes the principal TM mode.
The expansion coefficients in \eqref{eq:Einc1expand} are determined
by equating \eqref{eq:Einc1} and \eqref{eq:Einc1expand} and
invoking the orthogonality of the Floquet modes: 
\begin{gather}
  \iint_{U'} \bigl(a_1\one \e\one_{100} + a_2\one \e\one_{200}\bigr) \cross
  (\h\one_{100})^* \bdot \z \, \d A 
  =
  \iint_{U'} \tauhat_1\inc e^{-j\k_1\inc \bdot \vecrho} \cross
  \bigl(\h\one_{100}\bigr)^* \bdot \z \, \d A \nonumber \\
                                %
  P\one_{100} a\one_{100} =
  \iint_{U'} \tauhat_1\inc \cross 
  \bigl[
    \bigl(Y\one_{100}\bigr)^* 
    \z \cross \bigl(c\one_{100}\bigr)^* \t_{100})
  \bigr] \bdot \z \, \d A 
  =\bigl(Y\one_{100}\bigr)^* \bigl(c\one_{100}\bigr)^* A
  \tauhat_1\inc \bdot \t_{100}
\end{gather}
so that 
\begin{align}
  a\one_{100} &= \frac{\bigl(Y\one_{100}\bigr)^* %
    \bigl(c\one_{100}\bigr)^* A   \tauhat_1\inc \bdot \t_{100}}%
  {P_{100}}
  = 
  \frac{\bigl(Y\one_{100}\bigr)^* %
    \bigl(c\one_{100}\bigr)^* A   \tauhat_1\inc \bdot \t_{100}}%
  {\abs{c\one_{100}}^2 \bigl(Y\one_{100}\bigr)^* A} 
  = \frac{\tauhat_1\inc \bdot \t_{100}}{c\one_{100}}
\end{align}
A similar derivation shows that
\begin{equation}
  a\one_{200} = \frac{\tauhat_1\inc \bdot \t_{200}}{c\one_{100}}
\end{equation}

\subsubsection{Region~1 Scattered Fields}

The fields scattered by the FSS into Region~$1$ can now be written as
\begin{align}
  \E\scat(0,0,z_1)
  &=
  b\one_{100} \e\one_{100} + b\one_{200} \bigl(\e\one_{200} +
   j \z c_{200}\one \beta_{00} / \gamma_{00}\one  \bigr)
   \label{eq:escat11}
\end{align}
where $b\one_{100}$ and $b\one_{200}$ are determined from the
scattering relation \eqref{eq:fssscat}.  

Suppose now that we wish to determine the components of the Region~1
scattered electric field along the two unit vectors 
\begin{subequations}
\begin{align}
  \tauhat_1\refl 
  & = \hhat_1\refl \cos \tau_1\refl + \vhat_1\refl \sin \tau_1\refl, \\
  \sigmahat_1\refl 
  & = - \hhat_1\refl \sin \tau_1\refl + 
  \vhat_1\refl \cos \tau_1\refl,
\end{align}
\end{subequations}
so that we may write
\begin{equation}
  \E\refl = b_\tau\one \tauhat_1\refl + b_\sigma\one \sigmahat_1\refl.
\end{equation}
The coefficients are easily found by dotting the corresponding unit
vectors with the scattered field from \eqref{eq:escat11}:
\begin{equation}
  b_\tau\one = \tauhat_1\refl \bdot   \E\scat(0,0,z_1), \quad
  b_\sigma\one = \sigmahat_1\refl \bdot   \E\scat(0,0,z_1).
\end{equation}
We will refer to these coeffcients as partial polarization reflection
coefficients and use the notation
\begin{equation}
  R^{11}(\tau_1\refl,\tau_1\inc) \equiv b_\tau\one, \quad
  R^{11}\left(\tau_1\refl+\piotwo,\tau_1\inc\right) \equiv b_\sigma\one.
\end{equation}



\subsubsection[Region~\textit{N}~Scattered Fields]{Region~\boldmath$N$\unboldmath~Scattered Fields}

The fields scattered by the FSS into Region~$N$ can now be written as
\begin{align}
  \E\scat(0,0,z_{N-1})
  &=
  b\enn_{100} \e\enn_{100} + b\enn_{200} \bigl(\e\enn_{200} -
   j \z c_{200}\enn \beta_{00} / \gamma_{00}\enn  \bigr)
   \label{eq:escatN1}
\end{align}
where $b\enn_{100}$ and $b\enn_{200}$ are determined from the
scattering relation \eqref{eq:fssscat}.  

Suppose that we wish to determine the components of the Region~$N$
scattered electric field along the two unit vectors 
\begin{subequations}
\begin{align}
  \tauhat_N\refl 
  & = \hhat_N\refl \cos \tau_N\refl + \vhat_N\refl \sin \tau_N\refl, \\
  \sigmahat_N\refl 
  & = - \hhat_N\refl \sin \tau_N\refl + 
  \vhat_N\refl \cos \tau_N\refl,
\end{align}
\end{subequations}
so that we may write
\begin{equation}
  \E\refl = b_\tau\enn \tauhat_N\refl + b_\sigma\enn \sigmahat_N\refl.
\end{equation}
The coefficients are easily found by dotting the corresponding unit
vectors with the scattered field from \eqref{eq:escatN1}:
\begin{equation}
  b_\tau\enn = \tauhat_N\refl \bdot   \E\scat(0,0,z_{N-1}), \quad
  b_\sigma\enn = \sigmahat_N\refl \bdot   \E\scat(0,0,z_{N-1}).
\end{equation}
We will refer to these coeffcients (after renormalization)
as partial polarization transmission coefficients and use the notation
\begin{equation}
  T^{21}(\tau_N\refl,\tau_1\inc) \equiv 
  \sqrt{\frac{\eta_1\cos\theta_N}{\eta_N\cos\theta_1}} b_\tau\enn, \quad
  T^{21}\left(\tau_N\refl+\piotwo,\tau_1\inc\right) \equiv 
    \sqrt{\frac{\eta_1\cos\theta_N}{\eta_N\cos\theta_1}} b_\sigma\enn.
\end{equation}
With this normalization the power conservation equation becomes:
\begin{equation}
  \abs{R^{11}(\tau_1\refl,\tau_1\inc)}^2 +
  \abs{R^{11}\left(\tau_1\refl+\piotwo,\tau_1\inc\right)}^2 +
  \abs{T^{21}(\tau_N\refl,\tau_1\inc)}^2 +
  \abs{T^{21}\left(\tau_N\refl+\piotwo,\tau_1\inc\right)}^2
  \leq 1
\end{equation}

\subsection[Region \textit{N} Incidence]{Region~$\boldsymbol{N}$ Incidence}

Consider a plane wave incident upon the FSS from Region~$N$:
\begin{equation}
  \E\inc = \tauhat_N\inc e^{-j\k_N\inc \bdot (\r - \z z_{N-1})}, 
  \quad z>z_{N-1}
  \label{eq:EincN}
\end{equation}
where
\begin{equation}
  \tauhat_N\inc = \hhat_N\inc \cos \tau_N\inc + \vhat_N\inc \sin \tau_N\inc.
\end{equation}
In order to find the fields scattered from the FSS by the incident
field in \eqref{eq:EincN}, we express the incident field as the sum of
a TE and TM wave:
\begin{equation}
  (\x\x+\y\y) \bdot \E\inc = a_1\enn \e\enn_{100} + a_2\enn \e\enn_{200}
  \label{eq:EincNexpand}
\end{equation}
where it will be recalled that a subscript of $(100)$ denotes the
principal TE mode and $(200)$ denotes the principal TM mode.
The expansion coefficients in \eqref{eq:EincNexpand} are determined
by equating \eqref{eq:EincN} and \eqref{eq:EincNexpand} in the
$z=z_{N-1}$ plane and
invoking the orthogonality of the Floquet modes: 
\begin{gather}
  \iint_{U'} \bigl(a_1\enn \e\enn_{100} + a_2\enn \e\enn_{200}\bigr) \cross
  (\h\enn_{100})^* \bdot \z \, \d A 
  =
  \iint_{U'} \tauhat_N\inc e^{-j\k_N\inc \bdot \vecrho} \cross
  \bigl(\h\enn_{100}\bigr)^* \bdot \z \, \d A \nonumber \\
                                %
  P\enn_{100} a\enn_{100} =
  \iint_{U'} \tauhat_N\inc \cross 
  \bigl[
    \bigl(Y\enn_{100}\bigr)^* 
    \z \cross \bigl(c\enn_{100}\bigr)^* \t_{100})
  \bigr] \bdot \z \, \d A 
  =\bigl(Y\enn_{100}\bigr)^* \bigl(c\enn_{100}\bigr)^* A
  \tauhat_N\inc \bdot \t_{100}
\end{gather}
so that 
\begin{align}
  a\enn_{100} &= \frac{\bigl(Y\enn_{100}\bigr)^* %
    \bigl(c\enn_{100}\bigr)^* A   \tauhat_N\inc \bdot \t_{100}}%
  {P\enn_{100}}
  = 
  \frac{\bigl(Y\enn_{100}\bigr)^* %
    \bigl(c\enn_{100}\bigr)^* A   \tauhat_N\inc \bdot \t_{100}}%
  {\abs{c\enn_{100}}^2 \bigl(Y\enn_{100}\bigr)^* A} 
  = \frac{\tauhat_N\inc \bdot \t_{100}}{c\enn_{100}}
\end{align}
A similar derivation shows that
\begin{equation}
  a\enn_{200} = \frac{\tauhat_N\inc \bdot \t_{200}}{c\enn_{100}}
\end{equation}

\subsubsection[Region~\textit{N} Scattered Fields]%
{Region~$\boldsymbol{N}$ Scattered Fields}

The fields scattered by the FSS into Region~$N$ can now be written as
\begin{align}
  \E\scat(0,0,z_{N-1})
  &=
  b\enn_{100} \e\enn_{100} + b\enn_{200} \bigl(\e\enn_{200} -
   j \z c_{200}\enn \beta_{00} / \gamma_{00}\enn  \bigr)
   \label{eq:escatNN}
\end{align}
where $b\enn_{100}$ and $b\enn_{200}$ are determined from the
scattering relation \eqref{eq:fssscat}.  

Suppose now that we wish to determine the components of the Region~$N$
scattered electric field along the two unit vectors 
\begin{subequations}
\begin{align}
  \tauhat_N\refl 
  & = \hhat_N\refl \cos \tau_N\refl + \vhat_N\refl \sin \tau_N\refl, \\
  \sigmahat_N\refl 
  & = - \hhat_N\refl \sin \tau_N\refl + 
  \vhat_N\refl \cos \tau_N\refl,
\end{align}
\end{subequations}
so that we may write
\begin{equation}
  \E\refl = b_\tau\enn \tauhat_N\refl + b_\sigma\enn \sigmahat_N\refl.
\end{equation}
The coefficients are easily found by dotting the corresponding unit
vectors with the scattered field from \eqref{eq:escatNN}:
\begin{equation}
  b_\tau\enn = \tauhat_N\refl \bdot   \E\scat(0,0,z_{N-1}), \quad
  b_\sigma\enn = \sigmahat_N\refl \bdot   \E\scat(0,0,z_{N-1}).
\end{equation}
We will refer to these coeffcients as partial polarization reflection
coefficients and use the notation
\begin{equation}
  R^{22}(\tau_N\refl,\tau_N\inc) \equiv b_\tau\enn, \quad
  R^{22}\left(\tau_N\refl+\piotwo,\tau_N\inc\right) \equiv b_\sigma\enn.
\end{equation}



\subsubsection{Region~1 Scattered Fields}

The fields scattered by the FSS into Region~$1$ can now be written as
\begin{align}
  \E\scat(0,0,z_{1})
  &=
  b\one_{100} \e\one_{100} + b\one_{200} \bigl(\e\one_{200} +
   j \z c_{200}\one \beta_{00} / \gamma_{00}\one  \bigr)
   \label{eq:escat1N}
\end{align}
where $b\one_{100}$ and $b\one_{200}$ are determined from the
scattering relation \eqref{eq:fssscat}.  

Suppose that we wish to determine the components of the Region~$1$
scattered electric field along the two unit vectors 
\begin{subequations}
\begin{align}
  \tauhat_1\refl 
  & = \hhat_1\refl \cos \tau_1\refl + \vhat_1\refl \sin \tau_1\refl, \\
  \sigmahat_1\refl 
  & = - \hhat_1\refl \sin \tau_1\refl + 
  \vhat_1\refl \cos \tau_1\refl,
\end{align}
\end{subequations}
so that we may write
\begin{equation}
  \E\refl = b_\tau\one \tauhat_1\refl + b_\sigma\one \sigmahat_1\refl.
\end{equation}
The coefficients are easily found by dotting the corresponding unit
vectors with the scattered field from \eqref{eq:escat1N}:
\begin{equation}
  b_\tau\one = \tauhat_1\refl \bdot   \E\scat(0,0,z_{1}), \quad
  b_\sigma\one = \sigmahat_1\refl \bdot   \E\scat(0,0,z_1).
\end{equation}
We will refer to these coeffcients (after renormalization)
as partial polarization transmission coefficients and use the notation
\begin{equation}
  T^{12}(\tau_1\refl,\tau_N\inc) \equiv 
  \sqrt{\frac{\eta_N\cos\theta_1}{\eta_1\cos\theta_N}} b_\tau\one, \quad
  T^{12}\left(\tau_1\refl+\piotwo,\tau_N\inc\right) \equiv 
    \sqrt{\frac{\eta_N\cos\theta_1}{\eta_1\cos\theta_N}} b_\sigma\one.
\end{equation}
With this normalization the power conservation equation 
becomes:
\begin{equation}
  \abs{R^{22}(\tau_N\refl,\tau_N\inc)}^2 +
  \abs{R^{22}\left(\tau_N\refl+\piotwo,\tau_N\inc\right)}^2 +
  \abs{T^{12}(\tau_1\refl,\tau_N\inc)}^2 +
  \abs{T^{12}\left(\tau_1\refl+\piotwo,\tau_N\inc\right)}^2
  \leq 1
\end{equation}

\subsection{Scattered Field Axial Ratio}
Let us assume that for either Region~1 or Region~$N$, the scattered
travelling wave coefficients $b_1$ and $b_2$ for the principal TE and
TM modes have been determined.  In order to compute the axial ratio,
we need to define two orthogonal directions $\uhat$ and $\what$, both
of which are orthogonal to the direction of propagation $\khat$.  
The scattered electric field vector will be expressed as components along
the $\uhat$ and $\what$ unit vectors:
\begin{equation}
  \E\refl = b_1 \e_1 + b_2 (\e_2 + \z e_{2z}) =
  \uhat E_u + \what E_w.
  \label{eq:Edecompose}
\end{equation}
Once the latter decomposition is found, we can calculate the linear
polarization rato
\begin{equation}
  P = E_w / E_u
  \label{eq:Pdef}
\end{equation}
and then the circular polarization ratio
\begin{equation}
  Q = \frac{1 - jP}{1 + jP}.
\end{equation}
The axial ratio is found from $Q$ using
\begin{equation}
  A = \frac{1+ \abs{Q}}{\left\lvert 1 - \abs{Q}\right\rvert}
\end{equation}
We choose $\uhat = \t_1 = \z \cross \vecbeta_{00}$ and 
\begin{align}
  \what &= \khat \cross \uhat \nonumber \\
  &= \frac{\beta_{00} \betahat_{00} \pm j \gamma_{00}\z}{k}
  \cross (\z\cross \betahat_{00}) \nonumber \\
  &= \frac{1}{k} 
  \left[
    \beta_{00} \betahat_{00} \cross (\z\cross \betahat_{00})
    \pm j \gamma_{00}\z
  \cross (\z\cross \betahat_{00}) 
  \right] \nonumber \\
  &= \frac{1}{k} 
  \left[
    \beta_{00} \z
    \mp j \gamma_{00} \betahat_{00}) 
  \right] 
  =
  \frac{1}{k} 
  \left[
    \beta_{00} \z
    \mp j \gamma_{00} \t_2) 
  \right] 
\end{align}
where we have dispensed with the region designators, the top sign
is selected for Region~$1$, and the bottom sign for Region~$N$.
With these explicit expressions for $\uhat$ and $\what$ in hand, we
may determine $E_u$ and $E_w$ from \eqref{eq:Edecompose}:
\begin{align}
  E_u 
  &= \uhat \bdot 
  \left[b_1 \e_1 + b_2 
    \left(
      \e_2 \pm \z \frac{jc_2\beta_{00}}{\gamma_{00}}
    \right)
  \right] 
  = \t_1 \bdot 
  \left[b_1 \e_1 + b_2 
    \left(
      \e_2 \pm \z \frac{jc_2\beta_{00}}{\gamma_{00}}
    \right)
  \right] \nonumber \\
  &= b_1 c_1  \\
  E_w
  &= \what \bdot \left[b_1 \e_1 + b_2 
    \left(
      \e_2 \pm \z \frac{jc_2\beta_{00}}{\gamma_{00}}
    \right)
  \right]
  = 
  \frac{b_2}{k} 
  \left[
    \beta_{00} \z
    \mp j \gamma_{00} \t_2) 
  \right] 
  \bdot 
  \left(
    \e_2 \pm \z \frac{jc_2\beta_{00}}{\gamma_{00}}
  \right) \nonumber \\
  &=
  \frac{b_2}{k} 
  \left[
    \pm \frac{jc_2\beta_{00}^2}{\gamma_{00}}
    \mp j c_2 \gamma_{00} 
  \right]
  =
  \pm \frac{j c_2 b_2}{k\gamma_{00}}
  \left[
    \beta_{00}^2 - \gamma_{00}^2
  \right]
  =
  \pm \frac{jk c_2 b_2}{\gamma_{00}}.
\end{align}
The linear polarization ratio is then calculated 
from \eqref{eq:Pdef} as
\begin{equation}
  P = \frac{E_w}{E_u}
  =   \pm \frac{jk c_2 b_2}{b_1 c_1 \gamma_{00}}
  = \pm \frac{b_2}{b_1} \frac{\sqrt{Y_1}}{\sqrt{Y_2}} \frac{jk}{\gamma_{00}}
  = \pm \frac{b_2}{b_1} 
  \frac{\sqrt{\frac{\gamma_{00}}{jk\eta}}}%
  {\sqrt{\frac{jk}{\eta\gamma_{00}}}} 
  \frac{jk}{\gamma_{00}}
  = \pm \frac{b_2}{b_1}
\end{equation}
Thus, the linear polarization ratio is just the ratio of the TE/TM
scattered mode coefficients found directly from the GSM.


\newpage
\section{Performance Parameters Provided by PSS}

\subsection{Reflection Coefficients}

Various reflection quantities are available for the user to examine.
These are shown in Table~\ref{tab:refl}.
\begin{table}[htbp]
  \begin{center}
    \renewcommand{\arraystretch}{1.3}
    \begin{tabular}{|c|l|} \hline
      \bfseries Quantity & \bfseries User Input \\ \hline \hline
      $\abs{\matel{S}^{11}_{m,n}}$ & \verb_S11MAG(m,n)_ \\ \hline
      $\abs{\matel{S}^{11}_{m,n}}^2$ & \verb_S11MSQ(m,n)_ \\ \hline
      $20\log_{10}\abs{\matel{S}^{11}_{m,n}}$ & \verb_S11DB(m,n)_ \\ \hline
      $\frac{180}{\pi}\arg{\matel{S}^{11}_{m,n}}$ & \verb_S11DEG(m,n)_
      \\ \hline
      $\abs{\matel{S}^{22}_{m,n}}$ & \verb_S22MAG(m,n)_ \\ \hline
      $\abs{\matel{S}^{22}_{m,n}}^2$ & \verb_S22MSQ(m,n)_ \\ \hline
      $20\log_{10}\abs{\matel{S}^{22}_{m,n}}$ & \verb_S22DB(m,n)_ \\ \hline
      $\frac{180}{\pi}\arg{\matel{S}^{22}_{m,n}}$ & \verb_S22DEG(m,n)_
      \\ \hline
      $\abs{R^{11}(\tau_1\refl,\tau_1\inc)}$ & 
        \verb_R11MAG(_$\tau_1\refl,\tau_1\inc$\verb_)_ \\ \hline
      $\abs{R^{11}(\tau_1\refl,\tau_1\inc)}^2$ & 
        \verb_R11MSQ(_$\tau_1\refl,\tau_1\inc$\verb_)_ \\ \hline
      $20\log_{10}\abs{R^{11}(\tau_1\refl,\tau_1\inc)}$ & 
        \verb_R11DB(_$\tau_1\refl,\tau_1\inc$\verb_)_ \\ \hline
      $\frac{180}{\pi}\arg{R^{11}(\tau_1\refl,\tau_1\inc)}$ & 
        \verb_R11DEG(_$\tau_1\refl,\tau_1\inc$\verb_)_ \\ \hline
      $\abs{R^{22}(\tau_N\refl,\tau_N\inc)}$ & 
        \verb_R22MAG(_$\tau_N\refl,\tau_N\inc$\verb_)_ \\ \hline
      $\abs{R^{22}(\tau_N\refl,\tau_N\inc)}^2$ & 
        \verb_R22MSQ(_$\tau_N\refl,\tau_N\inc$\verb_)_ \\ \hline
      $20\log_{10}\abs{R^{22}(\tau_N\refl,\tau_N\inc)}$ & 
        \verb_R22DB(_$\tau_N\refl,\tau_N\inc$\verb_)_ \\ \hline
      $\frac{180}{\pi}\arg{R^{22}(\tau_N\refl,\tau_N\inc)}$ & 
        \verb_R22DEG(_$\tau_N\refl,\tau_N\inc$\verb_)_ \\ \hline
    \end{tabular}
    \caption[Reflection quantities available in the PSS program]%
    {Reflection quantities available in the PSS program and
      the nomenclature used in the input script to specify them. 
      $m$ and $n$ are integer mode indices, where $1$ and $2$ always
      denote the principal TE and TM modes, respectively. All
      tilt angles are specified in degrees, and the symbols
      \textsf{H} and \textsf{V} are available as synonyms
      for \textsf{0}  and \textsf{90} respectively.} 
    \label{tab:refl}
  \end{center}
\end{table}
\cleardoublepage
\newpage
\subsection{Transmission Coefficients}

Various transmission quantities are available for the user to examine.
These are shown in Table~\ref{tab:trans}.
\begin{table}[htbp]
  \begin{center}
    \renewcommand{\arraystretch}{1.3}
    \begin{tabular}{|c|l|} \hline
      \bfseries Quantity & \bfseries User Input \\ \hline \hline
      $\abs{\matel{S}^{21}_{m,n}}$ & \verb_S21MAG(m,n)_ \\ \hline
      $\abs{\matel{S}^{21}_{m,n}}^2$ & \verb_S21MSQ(m,n)_ \\ \hline
      $20\log_{10}\abs{\matel{S}^{21}_{m,n}}$ & \verb_S21DB(m,n)_ \\ \hline
      $\frac{180}{\pi}\arg{\matel{S}^{21}_{m,n}}$ & \verb_S21DEG(m,n)_
      \\ \hline
      $\abs{\matel{S}^{12}_{m,n}}$ & \verb_S12MAG(m,n)_ \\ \hline
      $\abs{\matel{S}^{12}_{m,n}}^2$ & \verb_S12MSQ(m,n)_ \\ \hline
      $20\log_{10}\abs{\matel{S}^{12}_{m,n}}$ & \verb_S12DB(m,n)_ \\ \hline
      $\frac{180}{\pi}\arg{\matel{S}^{12}_{m,n}}$ & \verb_S12DEG(m,n)_
      \\ \hline
      $\abs{T^{21}(\tau_N\refl,\tau_1\inc)}$ & 
        \verb_T21MAG(_$\tau_N\refl,\tau_1\inc$\verb_)_ \\ \hline
      $\abs{T^{21}(\tau_N\refl,\tau_1\inc)}^2$ & 
        \verb_T21MSQ(_$\tau_N\refl,\tau_1\inc$\verb_)_ \\ \hline
      $20\log_{10}\abs{T^{21}(\tau_N\refl,\tau_1\inc)}$ & 
        \verb_T21DB(_$\tau_N\refl,\tau_1\inc$\verb_)_ \\ \hline
      $\frac{180}{\pi}\arg{T^{21}(\tau_N\refl,\tau_1\inc)}$ & 
        \verb_T21DEG(_$\tau_N\refl,\tau_1\inc$\verb_)_ \\ \hline
      $\abs{T^{12}(\tau_1\refl,\tau_N\inc)}$ & 
        \verb_T12MAG(_$\tau_1\refl,\tau_N\inc$\verb_)_ \\ \hline
      $\abs{T^{12}(\tau_1\refl,\tau_N\inc)}^2$ & 
        \verb_T12MSQ(_$\tau_1\refl,\tau_N\inc$\verb_)_ \\ \hline
      $20\log_{10}\abs{T^{12}(\tau_1\refl,\tau_N\inc)}$ & 
        \verb_T12DB(_$\tau_1\refl,\tau_N\inc$\verb_)_ \\ \hline
      $\frac{180}{\pi}\arg{T^{12}(\tau_N\refl,\tau_N\inc)}$ & 
        \verb_T12DEG(_$\tau_1\refl,\tau_N\inc$\verb_)_ \\ \hline
    \end{tabular}
    \caption[Transmission quantities available in the PSS program]%
    {Transmission quantities available in the PSS program and
      the nomenclature used in the input script to specify them. 
      $m$ and $n$ are integer mode indices, where $1$ and $2$ always
      denote the principal TE and TM modes, respectively. All
      tilt angles are specified in degrees, and the symbols
      \textsf{H} and \textsf{V} are available as synonyms
      for \textsf{0}  and \textsf{90} respectively.} 
    \label{tab:trans}
  \end{center}
\end{table}

\cleardoublepage
\newpage


\subsection{Differential Reflection Coefficients}

Various ratios and differences of reflection quantities are
available for the user to examine. These are shown in Table~\ref{tab:drefl}.
\begin{table}[htbp]
  \begin{center}
    \renewcommand{\arraystretch}{1.3}
    \begin{tabular}{|c|l|} \hline
      \bfseries Quantity & \bfseries User Input \\ \hline \hline
      $\abs{\matel{S}^{11}_{m,m} / \matel{S}^{11}_{n,n}}$ & 
      \verb_DS11MAG(m,n)_ \\ \hline
      $\abs{\matel{S}^{11}_{m,m} / \matel{S}^{11}_{n,n}}^2$ & 
      \verb_DS11MSQ(m,n)_ \\ \hline
      $20\log_{10}\abs{\matel{S}^{11}_{m,m}/\matel{S}^{11}_{n,n}}$ & 
      \verb_DS11DB(m,n)_ \\ \hline
      $\frac{180}{\pi}\arg\left\{\matel{S}^{11}_{m,m} /
        \matel{S}^{11}_{n,n}\right\}$
      & \verb_DS11DEG(m,n)_   \\ \hline
      $\abs{\matel{S}^{22}_{m,m}/\matel{S}^{22}_{n,n}}$ & 
      \verb_DS22MAG(m,n)_ \\ \hline
      $\abs{\matel{S}^{22}_{m,m}/\matel{S}^{22}_{n,n}}^2$ & 
      \verb_DS22MSQ(m,n)_ \\ \hline
      $20\log_{10}\abs{\matel{S}^{22}_{m,m}/\matel{S}^{22}_{n,n}}$ & 
      \verb_DS22DB(m,n)_ \\ \hline
      $\frac{180}{\pi}\arg\left\{\matel{S}^{22}_{m,m} / 
        \matel{S}^{22}_{n,n}\right\}$
      & \verb_DS22DEG(m,n)_   \\ \hline
      $\abs{R^{11}(\tau_1,\tau_1) / 
        R^{11}(\tau_2,\tau_2)}$ & 
      \verb_DR11MAG(_$\tau_1,\tau_2$\verb_)_ \\ \hline
      $\abs{R^{11}(\tau_1,\tau_1) /
        R^{11}(\tau_2,\tau_2)}^2$ & 
      \verb_DR11MSQ(_$\tau_1,\tau_2$\verb_)_ \\ \hline
      $20\log_{10}\abs{R^{11}(\tau_1,\tau_1) / 
        R^{11}(\tau_2,\tau_2)}$ & 
      \verb_DR11DB(_$\tau_1\tau_2$\verb_)_ \\ \hline
      $\frac{180}{\pi}\arg\left\{R^{11}(\tau_1,\tau_1) / 
        R^{11}(\tau_2,\tau_2)\right\}$ & 
        \verb_DR11DEG(_$\tau_1,\tau_2$\verb_)_ \\ \hline
      $\abs{R^{22}(\tau_1,\tau_1) / 
        R^{22}(\tau_2,\tau_2)}$ & 
        \verb_DR22MAG(_$\tau_1,\tau_2$\verb_)_ \\ \hline
        $\abs{R^{22}(\tau_1,\tau_1) / 
          R^{22}(\tau_2,\tau_2)}^2$ & 
        \verb_DT22MSQ(_$\tau_1,\tau_2$\verb_)_ \\ \hline
      $20\log_{10}\abs{R^{22}(\tau_1,\tau_1) / 
        R^{22}(\tau_2,\tau_2)}$ & 
        \verb_DT22DB(_$\tau_1,\tau_2$\verb_)_ \\ \hline
      $\frac{180}{\pi}\arg\left\{R^{22}(\tau_1,\tau_1) / 
        R^{22}(\tau_2,\tau_2)\right\}$ & 
        \verb_DT22DEG(_$\tau_1,\tau_2$\verb_)_ \\ \hline
        Region $1$ scattered field axial ratio & 
        \verb_AR11DB(_$\tau_1\inc$\verb_)_ \\ \hline
        Region $N$ scattered field axial ratio & 
        \verb_AR22DB(_$\tau_N\inc$\verb_)_ \\ \hline
    \end{tabular}
    \caption[Differential reflection quantities available in the PSS
      program]%
    {Differential reflection quantities available in the PSS program
      and the nomenclature used in the input script to specify them. 
      $m$ and $n$ are integer mode indices, where $1$ and $2$ always
      denote the principal TE and TM modes, respectively. All
      tilt angles are specified in degrees, and the symbols
      \textsf{H} and \textsf{V} are available as synonyms
      for \textsf{0}  and \textsf{90} respectively.} 
    \label{tab:drefl}
  \end{center}
\end{table}

\cleardoublepage
\newpage


\subsection{Differential Transmission Coefficients}

Various ratios and differences of transmission quantities are
available for the user to examine. These are shown in Table~\ref{tab:dtrans}.
\begin{table}[h]
  \begin{center}
    \renewcommand{\arraystretch}{1.25}
    \begin{tabular}{|c|l|} \hline
      \bfseries Quantity & \bfseries User Input \\ \hline \hline
      $\abs{\matel{S}^{21}_{m,m} / \matel{S}^{21}_{n,n}}$ & 
      \verb_DS21MAG(m,n)_ \\ \hline
      $\abs{\matel{S}^{21}_{m,m} / \matel{S}^{21}_{n,n}}^2$ & 
      \verb_DS21MSQ(m,n)_ \\ \hline
      $20\log_{10}\abs{\matel{S}^{21}_{m,m}/\matel{S}^{21}_{n,n}}$ & 
      \verb_DS21DB(m,n)_ \\ \hline
      $\frac{180}{\pi}\arg\left\{\matel{S}^{21}_{m,m} /
        \matel{S}^{21}_{n,n}\right\}$
      & 
      \verb_DS21DEG(m,n)_   \\ \hline
      $\abs{\matel{S}^{12}_{m,m}/\matel{S}^{12}_{n,n}}$ & 
      \verb_DS12MAG(m,n)_ \\ \hline
      $\abs{\matel{S}^{12}_{m,m}/\matel{S}^{12}_{n,n}}^2$ & 
      \verb_DS12MSQ(m,n)_ \\ \hline
      $20\log_{10}\abs{\matel{S}^{12}_{m,m}/\matel{S}^{12}_{n,n}}$ & 
      \verb_DS12DB(m,n)_ \\ \hline
      $\frac{180}{\pi}\arg\left\{\matel{S}^{12}_{m,m} / 
        \matel{S}^{12}_{n,n}\right\}$
      & 
      \verb_S12DEG(m,n)_   \\ \hline
      $\abs{T^{21}(\tau_1,\tau_1) / 
        T^{21}(\tau_2,\tau_2)}$ & 
      \verb_DT21MAG(_$\tau_1,\tau_2$\verb_)_ \\ \hline
      $\abs{T^{21}(\tau_1,\tau_1) /
        T^{21}(\tau_2,\tau_2)}^2$ & 
      \verb_DT21MSQ(_$\tau_1,\tau_2$\verb_)_ \\ \hline
      $20\log_{10}\abs{T^{21}(\tau_1,\tau_1) / 
        T^{21}(\tau_2,\tau_2)}$ & 
      \verb_DT21DB(_$\tau_1,\tau_2$\verb_)_ \\ \hline
      $\frac{180}{\pi}\arg\left\{T^{21}(\tau_1,\tau_1) / 
        T^{21}(\tau_2,\tau_2)\right\}$ & 
        \verb_DT21DEG(_$\tau_1,\tau_2$\verb_)_ \\ \hline
      $\abs{T^{12}(\tau_1,\tau_1) / 
        T^{12}(\tau_2,\tau_2)}$ & 
        \verb_DT12MAG(_$\tau_1,\tau_2$\verb_)_ \\ \hline
        $\abs{T^{12}(\tau_1,\tau_1) / 
          T^{12}(\tau_2,\tau_2)}^2$ & 
        \verb_DT12MSQ(_$\tau_1,\tau_2$\verb_)_ \\ \hline
      $20\log_{10}\abs{T^{12}(\tau_1,\tau_1) / 
        T^{12}(\tau_2,\tau_2)}$ & 
        \verb_DT12DB(_$\tau_1,\tau_2$\verb_)_ \\ \hline
      $\frac{180}{\pi}\arg\left\{T^{12}(\tau_1,\tau_1) / 
        T^{12}(\tau_2,\tau_2)\right\}$ & 
        \verb_DT12DEG(_$\tau_1,\tau_2$\verb_)_ \\ \hline
        Region $N$ scattered field axial ratio & 
        \verb_AR21DB(_$\tau_1\inc$\verb_)_ \\ \hline
        Region $1$ scattered field axial ratio & 
        \verb_AR12DB(_$\tau_N\inc$\verb_)_ \\ \hline
    \end{tabular}
    \caption[Differential transmission quantities available in the PSS
      program]%
    {Differential transmission quantities available in the PSS program
       and the nomenclature used in the input script to specify them. 
       $m$ and $n$ are integer mode indices, where $1$ and $2$ always
       denote the principal TE and TM modes, respectively. All
       tilt angles are specified in degrees, and the symbols
       \textsf{H} and \textsf{V} are available as synonyms
      for \textsf{0}  and \textsf{90} respectively.} 
    \label{tab:dtrans}
  \end{center}
\end{table}

\cleardoublepage
\newpage



\addcontentsline{toc}{section}{References}
\bibliographystyle{ieeetr}
%\bibliography{/analysis0/s/simonp/bibtex/antcad}
\bibliography{e:/TeX/texmf-local/bibtex/bib/antcad.bib}

\end{document}







