\chapter{Green's Functions for Multiply Stratified Medium}
\label{chap:gfstratified}
In the previous chapter, a method was derived to efficiently evaluate the
potential Green's functions for a geometry consisting of the abutment
of two half-spaces under quasi-periodic boundary conditions.  This
chapter extends the results to handle an arbitrary
number of dielectric slabs. 

 The structure for which the potential Green's
functions are desired consists of $N$ layers, as shown in
Figure~\ref{fig:geom3}.
\begin{figure}[htbp]
  \begin{center} \footnotesize
    \leavevmode
    \pspicture(1,4.2)(12.9,10)
    \psset{nodesep=2.5pt}
                                % Begin Input Medium:
    \rput(1.5,7){\cstack{Layer \\1 \\ $(\mu_1,\epsilon_1)$}}
    \psline[linewidth=1pt](2.3,4.5)(2.3,8.5)  \uput[d](2.3,4.5){$z_1$}
    \pnode(2.3,8.5){Junction1}
    \rput[b](1.8,9.5){\rnode{Junction1label}{Junction~1}}
    \ncline[linewidth=0.5pt]{->}{Junction1label}{Junction1}
                                % Begin Layer 2:
    \rput(3.15,7){\cstack{Layer \\ 2 \\ $(\mu_2,\epsilon_2)$}}
    \psline[linewidth=1pt](4,4.5)(4,8.5)    \uput[d](4,4.5){$z_2$}
    \pnode(4,8.5){Junction2}
    \rput[b](4,9.5){\rnode{Junction2label}{Junction 2}}
    \ncline[linewidth=0.5pt]{->}{Junction2label}{Junction2}
    \pcline[linewidth=0.5pt,nodesep=1pt]{<->}(2.3,5.5)(4,5.5) \lput*{:U}{$h^{(2)}$}
                                % Begin Layer 3:
    \rput(5,7){\cstack{Layer \\ 3 \\ $(\mu_3,\epsilon_3)$}}
    \psline[linewidth=1pt](6,4.5)(6,8.5)    \uput[d](6,4.5){$z_3$}
    \pnode(6,8.5){Junction3}
    \rput[b](5.5,9){\rnode{Junction3label}{Junction 3}}
    \ncline[linewidth=0.5pt]{->}{Junction3label}{Junction3}
    \pcline[linewidth=0.5pt,nodesep=1pt]{<->}(4,5.5)(6,5.5) \lput*{:U}{$h^{(3)}$}
                                % Ellipses:
    \rput(6.5,6.5){\large$\boldsymbol{\cdots}$}
                                % Begin junction N-3:
    \psline[linewidth=1pt](7,4.5)(7,8.5)    \uput[d](7,4.5){$z_{N-3}$}
    \pnode(7,8.5){JunctionNsm2}
    \rput[b](6.5,9.5){\rnode{JunctionNsm2label}{Junction $N-3$}}
    \ncline[linewidth=0.5pt]{->}{JunctionNsm2label}{JunctionNsm2}
                                % Begin Layer N-2:
    \rput(8.125,7){\cstack{Layer \\ $N-2$ \\ $(\mu_{N-2},\epsilon_{N-2})$}}
    \psline[linewidth=1pt](9.25,4.5)(9.25,8.5)    \uput[d](9.25,4.5){$z_{N-2}$}  
    \pnode(9.25,8.5){JunctionNsm1}
    \rput[b](9.25,9){\rnode{JunctionNsm1label}{Junction $N-2$}}
    \ncline[linewidth=0.5pt]{->}{JunctionNsm1label}{JunctionNsm1}
    \pcline[linewidth=0.5pt,nodesep=1pt]{<->}(7,5.5)(9.25,5.5) \lput*{:U}{$h^{(N-2)}$}
                                % Begin Layer N-1:
    \rput(10.375,7){\cstack{Layer \\ $N-1$ \\ $(\mu_{N-1},\epsilon_{N-1})$}}
    \psline[linewidth=1pt](11.5,4.5)(11.5,8.5)    \uput[d](11.5,4.5){$z_{N-1}$}    
    \pnode(11.5,8.5){JunctionNs}
    \rput[b](11.5,9.5){\rnode{JunctionNslabel}{Junction $N-1$}}
    \ncline[linewidth=0.5pt]{->}{JunctionNslabel}{JunctionNs}
    \pcline[linewidth=0.5pt,nodesep=1pt]{<->}(9.25,5.5)(11.5,5.5) \lput*{:U}{$h^{(N-1)}$}
                                % Begin Output Medium:
    \rput(12.4,7){\cstack{Layer \\ $N$ \\ $(\mu_{N},\epsilon_{N})$}}
                                % Incident wave vector
%     \pcline[linewidth=1pt,nodesep=1pt]{->}(0.5679,6.0000)(2.3,5) \aput{:U}(0.3){$\khatinc$}
%     \psarc[linewidth=0.5pt]{<-}(2.3,5){1}{150}{180}
%     \uput{30pt}[170](2.3,5){$\thetainc$}
%     \pcline[linewidth=0.8pt,nodesep=0pt]{->}(2.3,5)(0.3,5) \aput[4pt]{0}(1){$-\z$}
    \endpspicture
  \end{center}
  \caption{The structure under consideration is a stack of $N \geq 2$
    dielectric layers.  Layers~1 and $N$ are semi-infinite in extent.
    For $2 \leq i \leq N-1$, layer~$i$ lies in the region $z_{i-1} < z
    < z_i$, is of thickness $h\eye = z_i - z_{i-1}$, and is characterized by
    permeability $\mu_i$ and permittivity $\epsilon_i$.}
  \label{fig:geom3}
\end{figure}
%
The structure is laterally invariant, with each dielectric layer being
homogeneous and isotropic.
Each layer $i=1,2, \ldots, N$ is characterized by a complex
permittivity $\epsilon_i$ and permeability $\mu_i$, each of which lies
either in the fourth quadrant of the complex plane, or on the real axis.  The
medium intrinsic wavenumbers are $k_i = \omega\sqrt{\mu_i\epsilon_i}$.


We will conntinue the use of the shorthand notation $\displaystyle \summn$ to denote
the double sum $\displaystyle \sum_{m=-\infty}^{\infty} \sum_{n=-\infty}^{\infty}.$


As in the Chapter~\ref{chap:mpgf} we are interested in determining the
potential Green's functions
$G^A_{xx}(\vecrho-\vecrho',z,z'),$ 
$G^F_{xx}(\vecrho-\vecrho',z,z'),$ 
$G^\Phi(\vecrho-\vecrho',z,z'),$ and $G^\Psi(\vecrho-\vecrho',z,z')$
 evaluated for both observation point
$z$ and source point $z'$ located at one of the interface
planes: $z=z'=z_s$, $s=1,2,\ldots,N-1.$  To accomplish this task we
will make use of the
transmission line Green's function (TLGF) formalism of \cite{mimo:97}.
Therefore, we must first determine the relationship between the
quasi-periodic Green's functions and those discussed in
\cite{mimo:97}. We investigate this relationship in the next section.

\section{The Discrete Spectrum of Quasi-Periodic Functions}
Let us suppose that we are given the complex-valued function $f:
\Realnum^2 \rightarrow \Complexnum$ and its two-dimensional Fourier
transform $\tilde{f}: \Realnum^2 \rightarrow \Complexnum.$ They are
obtained from each other by the transform relations
\begin{equation}
  \tilde{f}(\k) = \iint_{\Realnum^2} \!\! f(\vecrho) e^{j\k \bdot
  \vecrho} \,
  \d x \, \d y, \quad
  f(\vecrho) = \frac{1}{4\pi^2}\iint_{\Realnum^2} \!\! \tilde{f}(\k) e^{-j\k \bdot
  \vecrho} \,
  \d k_x \, \d k_y,
\end{equation}
where $\vecrho = \x x + \y y$ and $\k = \x k_x + \y k_y.$
We now define a quasi-periodic%
%
\footnote{``Quasi-periodic'' because of the 
  incremental phase shifts $\psi_1$ and $\psi_2$
  applied to the otherwise periodic function.}%
%
function $f_p$ as
\begin{equation}
  f_p(\vecrho) = \summn f(\vecrho - m\s_1 - n\s_2) e^{-j(m\psi_1+n\psi_2)}
\end{equation}
with Fourier transform $\tilde{f}_p$.
We are interested in determining the relationship between
$\tilde{f}_p$ and $\tilde{f}.$

Applying the transform integral to $f_p$ and interchanging the order
of summation and integration we obtain
\begin{align}
  \tilde{f}_p(\k) &= 
  \summn \iint_{\Realnum^2} f(\vecrho-m\s_1-n\s_2)
  e^{-j(m\psi_1+n\psi_2)} e^{j\k\bdot\vecrho}
  \, \d x \, \d y  \nonumber \\
  %
  &= \summn e^{-j(m\psi_1+n\psi_2)} \iint_{\Realnum^2} f(\vecrho)
  e^{j\k\bdot(\vecrho+m\s_1+n\s_2)}
  \, \d x \, \d y  \nonumber \\
  %
  &= \summn e^{-j(m\psi_1+n\psi_2)} 
  e^{j\k\bdot(m\s_1+n\s_2)}
  \iint_{\Realnum^2} f(\vecrho)
  e^{j\k\bdot\vecrho}
  \, \d x \, \d y  \nonumber \\
  %
  &= \tilde{f}(\k) \summn e^{-j(m\psi_1+n\psi_2)} 
  e^{j\k\bdot(m\s_1+n\s_2)}.  \label{eq:step1}
\end{align}
We now recall the reciprocal lattice vectors 
\begin{equation}
  \vecbeta_1 = \frac{2\pi}{A} \s_2\cross\z, \quad
  \vecbeta_2 = \frac{2\pi}{A} \z \cross \s_1,
\end{equation}
where $A = \norm{\s_1 \cross \s_2}$  is the unit cell area, and make
use of the property $\vecbeta_p \bdot \s_q = 2\pi \delta_{pq}, \; p,q
\in \{1,2\}.$  We also introduce the change of variables $\k = \xi
\vecbeta_1 + \eta \vecbeta_2 + \vecbeta_{00},$ where
\begin{equation}
  \vecbeta_{00} = \vecbeta_1\frac{\psi_1}{2\pi}  + \vecbeta_2\frac{\psi_2}{2\pi} 
\end{equation}
 so that \eqref{eq:step1} becomes
\begin{align}
  \tilde{f}_p(\xi\vecbeta_1 + \eta\vecbeta_2 + \vecbeta_{00}) 
  %
  &= \tilde{f}(\xi\vecbeta_1 + \eta\vecbeta_2 + \vecbeta_{00})  
  \summn e^{-j(m\psi_1+n\psi_2)} 
  e^{j(\xi\vecbeta_1 + \eta\vecbeta_2 + \vecbeta_{00})\bdot(m\s_1+n\s_2)}
  \nonumber \\
  %
  &= \tilde{f}(\xi\vecbeta_1 + \eta\vecbeta_2 + \vecbeta_{00})  
  \summn e^{-j(m\psi_1+n\psi_2)} 
  e^{j[m(2\pi\xi+\psi_1) + n(2\pi\eta+\psi_2)]}
   \nonumber \\
  %
  &= \tilde{f}(\xi\vecbeta_1 + \eta\vecbeta_2+\vecbeta_{00})  
  \summn 
  e^{j2\pi(m\xi + n\eta)} \nonumber \\
                                %
  &= \tilde{f}(\xi\vecbeta_1 + \eta\vecbeta_2 + \vecbeta_{00})  
  \summn 
  \delta(\xi - m) \, \delta(\eta-n) \nonumber \\
                                %
  &=   \summn 
  \tilde{f}(m\vecbeta_1 + n\vecbeta_2 + \vecbeta_{00})  
  \delta(\xi - m) \, \delta(\eta-n) \nonumber \\
  &=   \summn 
  \tilde{f}(\vecbeta_{mn})  
  \delta(\xi - m) \, \delta(\eta-n)
  \label{eq:step3}
\end{align}
where we have used the well-known Fourier series representation of a
train of delta functions and defined $\vecbeta_{mn} = m\vecbeta_1 +
n\vecbeta_2 + \vecbeta_{00}.$

The area element in the spectral domain is 
\begin{equation}
  \d k_x \, \d k_y = \d \xi \, \d \eta \norm{\vecbeta_1 \cross
  \vecbeta_2} = \frac{4\pi^2}{A} \, \d \xi \, \d \eta
\end{equation}
and the inversion integral applied to \eqref{eq:step3} becomes
\begin{align}
  f_p(\vecrho) 
                                %
  &= \frac{1}{4\pi^2} \iint_{\Realnum^2} \! \tilde{f}_p(\k) e^{-j\k \bdot
    \vecrho} \,  \d k_x \, \d k_y \nonumber \\
                                %
  &= \frac{1}{A} \iint_{\Realnum^2}
  \summn 
  \tilde{f}(\vecbeta_{mn})  
  \delta(\xi - m) \, \delta(\eta-n)
  e^{-j(\xi\vecbeta_1 + \eta\vecbeta_2 + \vecbeta_{00}) 
    \bdot \vecrho} \, \d \xi \, \d \eta \nonumber \\
                                %
  &= \frac{1}{A}   \summn 
  \tilde{f}(\vecbeta_{mn})  e^{-j\vecbeta_{mn} \bdot \vecrho}.
  \label{eq:spectrum}
\end{align}
Equation~\eqref{eq:spectrum} is the important, well-known result that the
quasi-periodic function $f_p$ can be recovered by sampling the spectrum of
the corresponding ``isolated'' function $f$ at locations determined by
the reciprocal lattice and impressed phasing. It allows us to
find the potential Green's functions for the quasi-periodic array of
point sources in the multilayered structure
from the spectral representations of the isolated-source Green's functions
presented in \cite{mimo:97}. 

\section{Potential Green's Functions (Electric Sources)}

\subsection{Magnetic Vector Potential}
According to \cite{mimo:97} the Fourier transform of the magnetic
vector potential Green's function (for an isolated source at $\vecrho'
= \0$) is
\begin{equation}
  \label{GA}
  \GtAxx(\k,z,z') = \frac{1}{j\omega}\Vi\TE(\k,z,z')
\end{equation}
where we have multiplied their result by $\mu_0$ to be consistent with
our definition of the potential Green's function, and $\Vi\TE(\k,z,z')$ is the
TE transmission line Green's function (TLGF) for the voltage observed at
$z$ due to a unit current source at $z'$, using an equivalent circuit
appropriate for a plane wave whose dependence on the transverse spatial
variables $x$ and $y$ is of the form $e^{-j\k \bdot \vecrho}$.
From Equation~\eqref{eq:spectrum} we then find that
\begin{equation}
  \label{eq:GA}
  \GAxx(\vecrho-\vecrho',z_s,z_s) = \frac{1}{j\omega A} \summn
  \Vi\TE(\vecbeta_{mn},z_s,z_s)   e^{-j\vecbeta_{mn} \bdot (\vecrho-\vecrho')}.
\end{equation}


\subsection{Scalar Electric Potential}
According to \cite{mimo:97} the Fourier transform of the so-called
``scalar potential kernel'' (for an isolated point charge associated
with a horizontal current source at $\vecrho'
= \0$) is
\begin{equation}
  \tilde{K}^{\Phi}(\k,z,z') = \frac{-j\omega\epsilon_0}{\k \bdot \k} 
  \left[\Vi\TE(\k,z,z') - \Vi\TM(\k,z,z')\right]
\end{equation}
where $\Vi\TM(\k,z,z')$ is the
TM transmission line Green's function (TLGF) for the voltage observed at
$z$ due to a unit current source at $z'$.  Dividing by $\epsilon_0$
to be consistent with our definition of the potential Green's function
and using Equation~\eqref{eq:spectrum} we obtain
\begin{equation}
  \label{eq:GPhi}
  \GPhi(\vecrho-\vecrho',z_s,z_s) = \frac{-j\omega}{A} \summn
  \frac{\Vi\TE(\vecbeta_{mn},z_s,z_s)-\Vi\TM(\vecbeta_{mn},z_s,z_s)}{\beta_{mn}^2}
  e^{-j\vecbeta_{mn} \bdot (\vecrho-\vecrho')}.
\end{equation}

\subsection{Evaluation of Transmission Line Green's Functions}
Referring to Equations~\eqref{eq:GA} and \eqref{eq:GPhi} we see that the
problem is reduced to evaluating the TLGFs $\Vi\TE$ and $\Vi\TM$.  The
equivalent transmission line circuit needed to accomplish this task is
shown in Figure~\ref{fig:tl}.  The equivalent circuit consists of a
series of cascaded transmission lines, each characterized by its
modal characteristic impedance $Z_{pmn}\eye$ and complex attenuation constant
$\gamma_{mn}\eye$. The termination impedances at the ends of the structure
are equal to the characteristic impedances of the respective
semi-infinite lines, and the entire structure is excited by a unit
current generator located at $z=z_s$.
\begin{figure}[tbp]
  \begin{center}
    \footnotesize
    \pspicture(-0.3,-1.9)(12,1.6)
    \psset{nodesep=0pt}
    \shuntload{0}{$Z_{pmn}^{(1)}$} 
    \zlabel{0}{$z_1$}
    \tlsection{0}{1.8}{2}
    \zlabel{1.8}{$z_2$}
    \tlsection{1.8}{3.5}{3}
    \tlsection{4.3}{6.4}{\ensuremath{s}}
    \zlabel{6.4}{$z_s$}
    \isource{6.4} \rput[l](6.45,0.35){$1\,\text{A}$}
    \tlsection{6.4}{8.7}{\ensuremath{s+1}}
    \tlsection{9.5}{11.6}{\ensuremath{N-1}}
    \shuntload{11.6}{$Z_{pmn}^{(N)}$}
    \zlabel{11.6}{$z_{N-1}$}
    \rput*(3.9,0.8){\huge$\boldsymbol{\cdots}$}
    \rput*(3.9,-0.8){\huge$\boldsymbol{\cdots}$}
    \rput*(9.1,0.8){\huge$\boldsymbol{\cdots}$}
    \rput*(9.1,-0.8){\huge$\boldsymbol{\cdots}$}
    \endpspicture
    \caption{Equivalent transmission line circuit used to find
    transmission line Green's functions.
    We set $p=1$ when evaluating $\Vi\TE$ and $p=2$ for
    $\Vi\TM$. Superscripted quantities in parentheses are region
    designators. $m$ and $n$ are modal indices associated with the
    transverse variation of the modal fields.}
    \label{fig:tl}
  \end{center}
\end{figure}
To calculate $\Vi\TE$, the modal impedances of the TE Floquet modes
are used in the equivalent circuit; for $\Vi\TM$ we employ the TM modal
impedances. The desired quantity is the voltage across the structure
at $z=z_s$.  Since we have a unit current generator at $z=z_s$ the
voltage there is equal to the total impedance at that point:
\begin{equation}
  \Vi^p(\vecbeta_{mn},z_s,z_s) = 
  Z_{pmn}^{\text{tot}}(z_s) = 
  \frac{1}{\frac{1}{\Zleft_{pmn}(z_s)} + \frac{1}{\Zright_{pmn}(z_s)}}
  =
  \frac{\Zleft_{pmn}(z_s) \, \Zright_{pmn}(z_s)}{\Zleft_{pmn}(z_s) + \Zright_{pmn}(z_s)},
\end{equation}
where $p$ takes on the value $1$ (TE) or $2$ (TM), 
and $\Zleft_{pmn}(z_s)$ and $\Zright_{pmn}(z_s)$ are the impedances seen looking
to the left and right, respectively, at $z=z_s$.  These latter
quantities are easily determined using the following recursive
formulas derived from elementary transmission line theory:
\begin{subequations}
  \begin{align}
    &\Zleft_{pmn}(z_1) = Z_{pmn}^{(1)} \\
    &\Zleft_{pmn}(z_i) = Z_{pmn}^{(i)} 
    \frac%
    {\Zleft_{pmn}(z_{i-1}) + Z_{pmn}\eye\tanh(\gamma_{mn}\eye h^{(i)})}%
    {Z_{pmn}\eye + \Zleft_{pmn}(z_{i-1})\tanh(\gamma_{mn}\eye h^{(i)})}, \quad i =
    2,3,\ldots,s \\
    &\Zright_{pmn}(z_{N-1}) = Z_{pmn}^{(N)} \\
    &\Zright_{pmn}(z_i) = Z_{pmn}^{(i+1)} 
    \frac%
    {\Zright_{pmn}(z_{i+1}) + Z_{pmn}^{(i+1)}\tanh(\gamma_{mn}^{(i+1)} h^{(i+1)})}%
    {Z_{pmn}^{(i+1)} + \Zright_{pmn}(z_{i+1})\tanh(\gamma_{mn}^{(i+1)} h^{(i+1)})}, \notag \\
    & \mkern300mu i = N-2,N-3,\ldots,s.
  \end{align}
\end{subequations}

\subsection{Series Acceleration}
At this point, we have expressed the Green's functions for the
magnetic vector potential and electric scalar potential as a pair of
modal series, as given in Equations~\eqref{eq:GA} and
\eqref{eq:GPhi}.  In order to accelerate the convergence of these
series we need to examine their asymptotic behavior.  Note that the
summands for both $\Vi\TE$ and $\Vi\TM$ involve $\Zleft_{pmn}(z_s)$
and $\Zright_{pmn}(z_s)$, both of which are of a similar form:
\begin{equation}
    \Zleft_{pmn}(z_s) = Z_{pmn}^{(s)} 
    \frac%
    {\Zleft_{pmn}(z_{s-1}) + Z_{pmn}\eye\tanh(\gamma_{mn}\eye h^{(s)})}%
    {Z_{pmn}^{(s)} + \Zleft_{pmn}(z_{s-1})\tanh(\gamma_{mn}^{(s)} h^{(s)})}
\end{equation}
The behavior of this quantity for large values of the spectral
variables $m$ and $n$ is determined by examining the following
asymptotic representation of the hyperbolic tangent function:
\begin{align}
  \tanh x &= \frac{1-e^{-2x}}{1+e^{-2x}}
  \sim (1-e^{-2x}) \times (1 - e^{-2x} + e^{-4x} - e^{-6x} +
  \cdots) \notag \\
  &= 1 - 2e^{-2x} + 2e^{-4x} - 2e^{-6x} + \cdots  
\end{align}
from which we see that the hyperbolic tangent function rapidly
approaches unity for large arguments.  In fact, one can approximate
$\tanh x \approx 1$ for $ x > 7$ with an error less than $2\times
10^{-6}.$ Therefore, the large argument approximation to the
transmission line Green's function  (assuming that the layers on each
side of the source plane are of nonvanishing thickness) is
\begin{equation}
  \Vi^p(\vecbeta_{mn},z_s,z_s) \sim
  \frac{Z_{pmn}^{(s)} \, Z_{pmn}^{(s+1)}}{Z_{pmn}^{(s)} + Z_{pmn}^{(s+1)}}
\end{equation}
which, when substituted into Equations~\eqref{eq:GA} and \eqref{eq:GPhi}
results in exactly the same formulas as presented for the two-layer
Green's functions of Chapter~\ref{chap:mpgf}.  We conclude that the
asymptotic behavior for the Green's functions is exactly the same as
for a two-layer structure consisting of the regions on either side of
the source plane, infinitely extended.  The acceleration technique is
thus the same as that presented in Chapter~\ref{chap:mpgf}.

The source-plane potential Green's functions can now be written in
their final, accelerated form as
\begin{subequations}
  \label{eq:finalEGF}
  \begin{align}
    G^A_{xx}(\vecrho-\vecrho',z_s,z_s) &=  
    \tilde{\mu}
    \biggl\{
    \Sigma_{m1} + 
    \frac{u}{4\pi}
    \left[
      \Sigma_{S1} + \frac{c_3({\scriptstyle
          \mu_s,\epsilon_s,\mu_{s+1},\epsilon_{s+1}})}{u^2} \Sigma_{S2}
    \right]
    \biggr\} \\
    G^\Phi(\vecrho-\vecrho',z_s,z_s) &=
    \frac{1}{\bar{\epsilon}}
    \biggl\{
    \Sigma_{m2} +
    \frac{u}{4\pi}
    \left[
      \Sigma_{S1} + \frac{d_3({\scriptstyle
          \mu_s,\epsilon_s,\mu_{s+1},\epsilon_{s+1}})}{u^2} \Sigma_{S2}
    \right]
    \biggr\} 
  \end{align}
\end{subequations}
where
\begin{subequations}
  \begin{align}
    \Sigma_{m1} &= 
    \frac{1}{2A} \summn
    \left[
      \frac{2\Vi\TE({\vecbeta_{mn},z_s,z_s})}{j\omega\tilde{\mu}} 
      - \frac{1}{\kappa_{mn}} -
      \frac{c_3({\scriptstyle\mu_s,\epsilon_s,\mu_{s+1},\epsilon_{s+1}})}{\kappa_{mn}^3}
    \right]  e^{-j\vecbeta_{mn}\bdot(\vecrho-\vecrho')} \\
                                %
    \Sigma_{m2} &= 
    \frac{1}{2A} \summn
    \Biggl(
    2j\omega\bar{\epsilon}
    \frac{\Vi\TM({\vecbeta_{mn},z_s,z_s})-\Vi\TE({\vecbeta_{mn},z_s,z_s})}%
    {\beta_{mn}^2}  \nonumber \\
    & \mbox{} \mkern 200mu - \frac{1}{\kappa_{mn}} -
    \frac{d_3({\scriptstyle\mu_s,\epsilon_s,\mu_{s+1},\epsilon_{s+1}})}{\kappa_{mn}^3}
    \Biggr)  e^{-j\vecbeta_{mn}\bdot(\vecrho-\vecrho')} 
  \end{align}
  \label{eq:finalmodalelectric}
\end{subequations}
and
\begin{equation}
  \tilde{\mu} =
  \frac{2\mu_s\mu_{s+1}}{\mu_s+\mu_{s+1}},
  \quad
  \bar{\epsilon} = \frac{\epsilon_s + \epsilon_{s+1}}{2}.
\end{equation}
The symbols $c_3$ and $d_3$ used above are defined in Equations~\ref{eq:cdef}
and \ref{eq:ddef}.






\section{Potential Green's Functions (Magnetic Sources)}

\subsection{Electric Vector Potential}
Using the definitions from Chapter~\ref{chap:fund}, we can obtain the
expression for the electric vector potential via duality from that of
the magnetic vector potential:
\begin{equation}
  \label{eq:GF}
  \GFxx(\vecrho-\vecrho',z,z') = \frac{-1}{j\omega A} \summn
  \Iv\TM(\vecbeta_{mn},z,z')   e^{-j\vecbeta_{mn} \bdot (\vecrho-\vecrho')}.
\end{equation}
where $\Iv\TM(\vecbeta_{mn},z,z')$ is the transmission line current at
$z'$ due to a unit series voltage source located at $z'$ for the TM
equivalent circuit.

\subsection{Scalar Magnetic Potential}
Using duality to transform the scalar electric potential we obtain
\begin{equation}
  \label{eq:GPsi}
  \GPsi(\vecrho-\vecrho',z,z') = \frac{-j\omega}{A} \summn
  \frac{\Iv\TM(\vecbeta_{mn},z,z')-\Iv\TE(\vecbeta_{mn},z,z')}{\beta_{mn}^2}
  e^{-j\vecbeta_{mn} \bdot (\vecrho-\vecrho')}.
\end{equation}


\subsection{Evaluation of Transmission Line Green's Functions}
The equivalent circuit for the magnetic source Green's function
differs from that of the electric sources.  In the integral equation
formulation, it is assumed that a perfectly conducting wall is
inserted at $z=z_s,$ with the sources impressed at both $z_s^-$ and
$z_s^+.$  Therefore, we will require two pairs of potential Green's
functions, denoted as $\GlFxx,$ $\GlPsi,$ $\GrFxx,$ and $\GrPsi.$  The
former two are due to sources impressed at $z_s^-$ (to the left of the ground
plane) and the latter two are for sources at $z_s^+$ (to the right of
the ground plane.)  The equivalent circuits for the left- and
right-looking sources are shown in Figure~\ref{fig:Mequivckt}.
\begin{figure}[tbp]
  \begin{center}
    \footnotesize
    \pspicture(-0.3,-1.7)(12,1.6)
    \psset{nodesep=0pt}
    \shuntload{0}{$Y_{q}^{(1)}$} 
    \zlabel{0}{$z_1$}
    \Ytlsection{0}{1.8}{2}
    \Ytlsection{2.6}{4.7}{\ensuremath{s}}
    \invertedvsource{4.7} \rput[l](4.75,0.35){$1\,\text{V}$}
    \zlabel{4.7}{$z_s$}
    \zlabel{6.4}{$z_s$}
    \vsource{6.4} \rput[l](6.45,0.35){$1\,\text{V}$}
    \Ytlsection{6.4}{8.7}{\ensuremath{s+1}}
    \Ytlsection{9.5}{11.6}{\ensuremath{N-1}}
    \shuntload{11.6}{$Y_{q}^{(N)}$}
    \zlabel{11.6}{$z_{N-1}$}
    \rput*(2.2,0.8){\huge$\boldsymbol{\cdots}$}
    \rput*(2.2,-0.8){\huge$\boldsymbol{\cdots}$}
    \rput*(9.1,0.8){\huge$\boldsymbol{\cdots}$}
    \rput*(9.1,-0.8){\huge$\boldsymbol{\cdots}$}
    \endpspicture
    \caption{Equivalent transmission line circuits used to find
    magnetic source transmission line Green's functions.
    We set $p=1$ when evaluating $\Iv\TE$ and $p=2$ for
    $\Iv\TM$.}
    \label{fig:Mequivckt}
  \end{center}
\end{figure}
Since we have unit voltage sources driving each equivalent circuit, it
is apparent that the source currents are just admittances, so that 
\begin{equation}
  \Ilv^{\mkern 4mu p}(\vecbeta_{mn},z_s,z_s) = \Yleft_{pmn}(z_s), \quad
  \Irv^{\mkern 4mu p}(\vecbeta_{mn},z_s,z_s) = \Yright_{pmn}(z_s).
\end{equation}
These admittances are easily found using the following recursive
formulas derived from elementary transmission line theory:
\begin{subequations}
  \begin{align}
    &\Yright_{pmn}(z_{N-1}) = Y_{pmn}^{(N)} \\
    &\Yright_{pmn}(z_i) = Y_{pmn}^{(i+1)} 
    \frac%
    {\Yright_{pmn}(z_{i+1}) + 
      Y_{pmn}^{(i+1)}\tanh(\gamma_{mn}^{(i+1)} h^{(i+1)})}%
    {Y_{pmn}^{(i+1)} + \Yright_{pmn}(z_{i+1})\tanh(\gamma_{mn}^{(i+1)}
      h^{(i+1)})}, 
    \notag \\
    & \mkern300mu i = N-2,N-3,\ldots,s.
  \end{align}
  \begin{align}
    &\Yleft_{pmn}(z_{1}) = Y_{pmn}\one, \\
    &\Yleft_{pmn}(z_i) = Y_{pmn}\eye 
    \frac%
    {\Yleft_{pmn}(z_{i-1}) + Y_{pmn}\eye \tanh(\gamma_{mn}\eye h\eye)}%
    {Y_{pmn}\eye + \Yleft_{pmn}(z_{i-1})\tanh(\gamma_{mn}\eye h\eye)}, \notag \\
    & \mkern300mu i = 2,3,\ldots,s.
  \end{align}
\end{subequations}

\subsection{Series Acceleration}
The asymptotic form of the magnetic source Green's functions  summands
are equal to twice the summand for a similar source radiating in a
homogeneous medium, with permittivity and permeability equal to either
Region~$s$ (for left-looking sources) or Region~$s+1$ (for
right-looking sources.)  Applying the acceleration techniques of
Chapter~\ref{chap:mpgf} then results in the final formulas
\begin{subequations}
  \begin{align}
    \GFxx(\vecrho-\vecrho',z_s,z_s)  &\equiv
    \GlFxx(\vecrho-\vecrho',z_s,z_s) +  \GrFxx(\vecrho-\vecrho',z_s,z_s) 
    \nonumber \\
                                %
    &=
    -\epsilon_0 
    \left[
      \Sigma'_{m1} + \frac{u\bar{\epsilon}}{\pi\epsilon_0} \Sigma_{s1}
      + \frac{c_3\ess \epsilon_s + c_3\spw\epsilon_{s+1}}%
      {2\pi u\epsilon_0} \Sigma_{s2} 
    \right]  \\
                                %
    \GPsi(\vecrho-\vecrho',z_s,z_s) &= 
    \GlPsi(\vecrho-\vecrho',z_s,z_s) +  \GrPsi(\vecrho-\vecrho',z_s,z_s)  
    \nonumber \\
    &= \frac{1}{\mu_0}
    \left[
      \Sigma'_{m2} + \frac{u\mu_0}{\pi\tilde{\mu}} \Sigma_{s1}
      + \frac{\mu_0}{2\pi u} 
      \Biggl(
      \frac{d_3\ess}{\mu_s}+\frac{d_3\spw}{\mu_{s+1}}
      \Biggr)
      \Sigma_{s2}
    \right]
  \end{align}
\end{subequations}
where
\begin{subequations}
  \begin{align}
    & \Sigma'_{m1} = \nonumber \\
    & \frac{1}{A} \summn
    \left[
      \frac{\Yleft_{2mn}(z_s) + \Yright_{2mn}(z_s)}{j\omega\epsilon_0} 
      - \frac{2\bar{\epsilon}}{\epsilon_0} \kappa_{mn}^{-1}
      - \frac{\epsilon_s c_3\ess + \epsilon_{s+1}c_3\spw}{\epsilon_0} 
      \kappa_{mn}^{-3}
    \right]
    e^{-j\vecbeta_{mn} \bdot (\vecrho-\vecrho')}  \\ 
                                %
    & \Sigma'_{m2} = \nonumber \\
    & \frac{1}{A} \summn
    \Biggl\{
    \left(
      \frac{\textstyle\Yleft_{2mn}(z_s) + \Yright_{2mn}(z_s)}
      {j\omega\epsilon_0} k_0^2
      + j\omega\mu_0\left[\Yleft_{1mn}(z_s) + \Yright_{1mn}(z_s)\right] 
    \right) 
    \bigg/ {\beta_{mn}^2} \nonumber \\
                                %
    & \mkern 150mu \mbox{} - \frac{2\mu_0}{\tilde{\mu}} \kappa_{mn}^{-1}
    - \mu_0\biggl(\frac{d_3\ess}{\mu_s}+\frac{d_3\spw}{\mu_{s+1}}\biggr)
    \kappa_{mn}^{-3}
    \Biggr\}
    e^{-j\vecbeta_{mn} \bdot (\vecrho-\vecrho')}
  \end{align}
  \label{eq:finalmodalmagnetic}
\end{subequations}
and
\begin{equation}
    c_3\eye = c_3(\epsilon_i,\mu_i,\epsilon_i,\mu_i), \quad
    d_3\eye = d_3(\epsilon_i,\mu_i,\epsilon_i,\mu_i).
\end{equation}

\section{FFT Evaluation of the Modal Difference Series}
\label{sec:modalFFT}
The modal series in  \eqref{eq:sigmam1}, \eqref{eq:sigmam2},
\eqref{eq:finalEGF}, \eqref{eq:finalmodalelectric}, and
\eqref{eq:finalmodalmagnetic} 
represent extremely
smooth functions of $\vecrho-\vecrho'$, since their singularities have
been subtracted out.  Therefore, an efficient and accurate method of
evaluating them is to tabulate and then interpolate using a low-order
bivariate polynomial interpolation scheme, 
such as \cite[Eq.~25.2.7]{abst:72}.  Here we consider a method of
rapidly tabulating the functions over the unit cell using the Fast
Fourier Transform (FFT).

The modal series are all of the form
\begin{equation}
  f(\vecrho-\vecrho') = 
  \sum_{m=-\infty}^{\infty} \sum_{n=-\infty}^{\infty} 
  f_{(m,n)} e^{-j(\vecbeta_{00} + m\vecbeta_1 + n\vecbeta_2) \bdot (\vecrho-\vecrho')}.
  \label{eq:fsum}
\end{equation}
We introduce the change of variables
\begin{equation}
  \vecrho-\vecrho' = \xi_1 \s_1 + \xi_2 \s_2.
  \label{eq:xi1xi2}
\end{equation}
It is clear that for any points $\vecrho$ and $\vecrho'$ in the unit
cell, the difference vector $\vecrho-\vecrho'$ can be represented by 
Equation~\eqref{eq:xi1xi2} with $-1 \leq \xi_1, \xi_2 < 1$.  We can
further restrict evaluation of the series to the range 
$0 \leq \xi_1, \xi_2 < 1$ by making use of the translational formula
\begin{equation}
  f((\xi_1+m)\s_1 + (\xi_2+n)\s_2) = e^{-j(m\psi_1+n\psi_2)}
f(\xi_1\s_1 + \xi_2\s_2)
\end{equation}
which holds for any integers $m$ and $n$.
Given $\vecrho-\vecrho'$ one can easily determine $\xi_1$ and $\xi_2$
using 
\begin{equation}
  \xi_i = \frac{1}{2\pi} \vecbeta_i \bdot (\vecrho-\vecrho'), 
  \quad i=1,2.  
\end{equation}
With \eqref{eq:xi1xi2}, the series in \eqref{eq:fsum}
  becomes
\begin{align}
  f(\xi_1\s_1+\xi_2\s_2) 
  &= 
  e^{-j(\xi_1\psi_1 + \xi_2\psi_2)}
  \sum_{m=-\infty}^{\infty} \sum_{n=-\infty}^{\infty} 
  f_{(m,n)} e^{-j2\pi(m\xi_1 + n\xi_2)}.
  %
\end{align}
We now assume that the summand is nonzero only for $-\frac{M}{2} \leq m \leq
\frac{M}{2}-1$ and $-\frac{N}{2} \leq n \leq
\frac{N}{2}-1$, where $M$ and $N$ are some convenient integer powers of $2$.
Then
\begin{align}
  f(\xi_1\s_1+\xi_2\s_2) 
  &= 
  e^{-j(\xi_1\psi_1 + \xi_2\psi_2)}
  \sum_{m=-\frac{M}{2}}^{\frac{M}{2}-1} \; \sum_{n=-\frac{N}{2}}^{\frac{N}{2}-1} 
  f_{(m,n)} e^{-j2\pi(m\xi_1 + n\xi_2)} \nonumber \\
  %
  &= 
  e^{-j(\xi_1\psi_1 + \xi_2\psi_2)}
  \sum_{m=0}^{M-1} \sum_{n=0}^{N-1} 
  f_{(m-\frac{M}{2},n-\frac{N}{2})} 
  e^{-j2\pi[(m-\frac{M}{2})\xi_1 + (n-\frac{N}{2})\xi_2]} \nonumber \\
  %
  &= 
  e^{-j[\xi_1(\psi_1-M\pi) + \xi_2(\psi_2-N\pi)]}
  \sum_{m=0}^{M-1} \sum_{n=0}^{N-1} 
  f_{(m-\frac{M}{2},n-\frac{N}{2})} e^{-j2\pi(m\xi_1 + n\xi_2)}.
  %
\end{align}
Finally, we restrict evaluation to the set of points
\begin{align*}
\xi_1 &= p / M, \; p = 0,1,2,\ldots, M-1, &
\xi_2 &= q / N, \; q = 0,1,2,\ldots, N-1
\end{align*}
so that the expression for the sum becomes
\begin{multline}
  f\left(\frac{p}{M} \s_1 + \frac{q}{N} \s_2\right) 
  = \\
  e^{j[p(\pi-\psi_1/M) + q(\pi-\psi_2/N)]}
  \sum_{m=0}^{M-1} \sum_{n=0}^{N-1} 
  f_{(m-\frac{M}{2},n-\frac{N}{2})} e^{-j2\pi(m p/M + n q/N)}, \\
  p = 0,1,2,\ldots, M-1, \quad q = 0,1,2,\ldots, N-1.
\label{eq:dft}
\end{multline}
The double sum in \eqref{eq:dft} constitutes a two-dimensional
discrete Fourier transform (DFT).  It can be evaluated efficiently
for all desired $p$ and $q$ values via a single application of the
two-dimensional fast Fourier transform (FFT) using the facilities
built into Julia.    
