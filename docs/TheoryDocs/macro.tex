%%%%%%%%%%%%%%%%%%%%%% Begin macro definitions  %%%%%%%%%%%%%%%%%%%%%%%%%%%%%%%
%\usepackage{bm} % Correct bold math
\newcommand{\pssfss}{PSSFSS}
\renewcommand{\d}{\text{d}}  % for use in integrals
\newcommand{\Realnum}{\mathbb{R}}
\newcommand{\Complexnum}{\mathbb{C}}
\newcommand{\Integers}{\mathbb{Z}}
\newcommand{\Fourier}{\mathcal{F}}
\DeclareMathOperator{\RealOp}{Re}
\DeclareMathOperator{\sgn}{sgn}
\newcommand{\Real}[1]{\RealOp\left\{#1\right\}}
\DeclareMathOperator{\ImagOp}{Im}
\newcommand{\Imag}[1]{\ImagOp\left\{#1\right\}}
\newcommand{\diag}[1]{\text{diag}\left\{#1\right\}}
\newcommand{\0}{\boldsymbol{0}}
\newcommand{\A}{\boldsymbol{A}}
\newcommand{\E}{\boldsymbol{E}}
\newcommand{\e}{\boldsymbol{e}}
\renewcommand{\k}{\boldsymbol{k}}
\newcommand{\f}{\boldsymbol{f}}  % Basis function.
\newcommand{\ftf}{\boldsymbol{\tilde{f}}}  % Basis function Fourier transform.
\newcommand{\F}{\boldsymbol{F}}
\renewcommand{\H}{\boldsymbol{H}}
\newcommand{\h}{\boldsymbol{h}}
\newcommand{\hhat}{\rlap{$\boldsymbol{\hat{h}}$}\phantom{\boldsymbol{h}}}
\newcommand{\vhat}{\rlap{$\boldsymbol{\hat{v}}$}\phantom{\boldsymbol{v}}}
\newcommand{\J}{\boldsymbol{J}}
\newcommand{\Jt}{\boldsymbol{\tilde{J}}}
\newcommand{\I}{\boldsymbol{I}}
\newcommand{\K}{\boldsymbol{K}}
\newcommand{\Js}{\J_{\text{s}}}
\newcommand{\ftJs}{\boldsymbol{\tilde{J}}_{\text{s}}}
\renewcommand{\L}{\boldsymbol{L}}
\newcommand{\M}{\boldsymbol{M}}
\newcommand{\Ms}{\boldsymbol{M}_{\text{s}}}
\newcommand{\ftMs}{\boldsymbol{\tilde{M}}_{\text{s}}}
\renewcommand{\r}{\boldsymbol{r}}
\newcommand{\x}{\boldsymbol{\hat{x}}}
\newcommand{\y}{\boldsymbol{\hat{y}}}
\newcommand{\z}{\boldsymbol{\hat{z}}}
\newcommand{\abs}[1]{\lvert#1\rvert}
\newcommand{\norm}[1]{\left\lVert#1\right\rVert}
\newcommand{\gradient}{\boldsymbol{\nabla}}
\newcommand{\cross}{\boldsymbol{\times}}
\newcommand{\bdot}{\boldsymbol{\cdot}}
\newcommand{\curl}{\boldsymbol{\nabla} \cross}
\newcommand{\divergence}{\boldsymbol{\nabla} \bdot}
\newcommand{\laplace}{\nabla^2}  % Scalar Laplacian Operator
\newcommand{\vlaplace}{\gradient^2}  % Vector Laplacian Operator
\newcommand{\qe}{q_{\text{e}}}  % Electric charge density
\newcommand{\qm}{q_{\text{m}}}  % Magnetic charge density
\newcommand{\union}{\bigcup}  % Magnetic charge density
\newcommand{\s}{\boldsymbol{s}}  % Direct lattice vector
\newcommand{\vecbeta}{\boldsymbol{\beta}}  % Reciprocal lattice vector
\newcommand{\betahat}{\boldsymbol{\hat{\beta}}}  % Reciprocal lattice unit vector
\newcommand{\vecrho}{\boldsymbol{\rho}}  % Reciprocal lattice vector
\newcommand{\colvec}[1]{\begin{bmatrix} #1 \end{bmatrix}}
%\newcommand{\colvec}[1]{\left[\begin{array}{c} #1 \end{array}\right]}
\newcommand{\TE}{^{\text{TE}}}
\newcommand{\TM}{^{\text{TM}}}
\newcommand{\mat}[1]{\boldsymbol{\mathcal{#1}}}
\newcommand{\matel}[1]{\mathcal{#1}}
\newcommand{\Icoef}{\matel{I}}  % Electric current expansion coefficient.
\newcommand{\Vcoef}{\matel{V}}  % Magnetic current expansion coefficient.
\newcommand{\one}{^{(1)}} % Region 1 designator
\newcommand{\two}{^{(2)}} % Region 2 designator
\newcommand{\eye}{^{(i)}} % Region i designator
\newcommand{\epw}{^{(i+1)}} % Region i+1 designator
\newcommand{\enn}{^{(N)}} % Region N designator
\newcommand{\ess}{^{(s)}} % Region s designator
\newcommand{\spw}{^{(s+1)}} % Region s+1 designator
\newcommand{\tmi}{^{(3-i)}} % Region 3-i designator
\newcommand{\three}{^{(3)}} % Region 3 designator
\newcommand{\four}{^{(4)}} % Region 4 designator
\newcommand{\transpose}{^{\text{\sffamily{T}}}}
\newcommand{\hermitian}{^{\text{\sffamily{H}}}}
\newcommand{\inc}{^{\text{inc}}}
\newcommand{\refl}{^{\text{r}}}
\newcommand{\trans}{^{\text{t}}}
\newcommand{\scat}{^{\text{sc}}}
\renewcommand{\t}{\boldsymbol{\hat{t}}}
\newcommand{\n}{\boldsymbol{\hat{n}}}
\newcommand{\pdx}[1]{\frac{\partial #1}{\partial x}}
\newcommand{\pdz}[1]{\frac{\partial #1}{\partial z}}
\newcommand{\pdu}[1]{\frac{\partial #1}{\partial u}}
\newcommand{\summn}{\sum_{m,n}}
\newcommand{\dyad}[1]{\boldsymbol{\bar{#1}}}
\newcommand{\Idemfactor}{\dyad{I}}
\newcommand{\GAxx}{G^A_{xx}}
\newcommand{\GtAxx}{\tilde{G}^{A}_{xx}}
\newcommand{\GFxx}{G^F_{xx}}
\newcommand{\GrFxx}{\raisebox{0pt}[0pt][0pt]{$\vecr{G}$}\vphantom{G}^F_{xx}}
\newcommand{\GlFxx}{\raisebox{0pt}[0pt][0pt]{$\vecl{G}$}\vphantom{G}^F_{xx}}
\newcommand{\GPhi}{{G}^{\Phi}}
\newcommand{\GtPhi}{{G}^{\Phi}}
\newcommand{\GPsi}{{G}^{\Psi}}
\newcommand{\GrPsi}{\raisebox{0pt}[0pt][0pt]{$\vecr{G}$}\vphantom{G}^\Psi}
\newcommand{\GlPsi}{\raisebox{0pt}[0pt][0pt]{$\vecl{G}$}\vphantom{G}^\Psi}
\newcommand{\rcp}{\r^{\text{c}+}} % Positive centroid r vector.
\newcommand{\rcm}{\r^{\text{c}-}} % Negative centroid r vector.
\newcommand{\rcpm}{\r^{\text{c}\pm}} % Plus or Negative centroid r vector.
\newcommand{\vecrhocp}{\vecrho^{\text{c}+}} % Positive centroid rho vector.
\newcommand{\vecrhocm}{\vecrho^{\text{c}-}} % Positive centroid rho vector.
\newcommand{\vecrhocpm}{\vecrho^{\text{c}\pm}} % Positive centroid rho vector.
\newcommand{\innerprod}[1]{\left< #1 \right>}
%\renewcommand{\iint}{\int \!\!\! \int}
\newcommand{\rcu}{\r^{\text{c}u}} % Centroid r vector for triangle ``u''.
\newcommand{\rhocu}{\rho^{\text{c}u}} % Centroid rho for triangle ``u''.
\renewcommand{\P}{\boldsymbol{P}}
\renewcommand{\l}{\boldsymbol{l}}
\newcommand{\lhat}{\boldsymbol{\hat{l}}}
\renewcommand{\u}{\boldsymbol{\hat{u}}}  % unit vector.
\newcommand{\is}{i_{\text{s}}}  % Source region designator
\renewcommand{\it}{i_{\text{t}}}  % Transmitted region designator
\newcommand{\eyes}{^{(\is)}} % Source Region designator
\newcommand{\eyet}{^{(\it)}} % Transmitted Region designator
% Vector accent pointing to left:
\newcommand{\vecl}[1]{\overset{\raisebox{-0.6ex}{\tiny$\boldsymbol{\leftarrow}$}}{#1}}  
% Vector accent pointing to right:
\newcommand{\vecr}[1]{\overset{\raisebox{-0.6ex}{\tiny$\mkern5mu\boldsymbol{\rightarrow}$}}{#1}}  
\newcommand{\Zright}{\vecr{Z}}
\newcommand{\Zleft}{\vecl{Z}}
%\newcommand{\Yleft}{\vecl{Y}}
%\newcommand{\Yright}{\vecr{Y}}
 \newcommand{\Yleft}%
     {\raisebox{0pt}[0pt][0pt]{\rlap{$\vecl{\phantom{Y}}$}}Y}  % TLGF
 \newcommand{\Yright}%
 {\raisebox{0pt}[0pt][0pt]{\rlap{$\vecr{\phantom{Y}}$}}Y}  % TLGF
\newcommand{\Vi}{V_{\text{i}}}  % TLGF
\newcommand{\Vv}{V_{\text{v}}}  % TLGF
\newcommand{\Ii}{I_{\text{i}}}  % TLGF
\newcommand{\Iv}{I_{\text{v}}}  % TLGF

 \newcommand{\Ilv}%
 {\raisebox{0pt}[0pt][0pt]{\rlap{$\vecl{I}$}}\phantom{I}_{\text{v}}}  % TLGF
 \newcommand{\Irv}%
 {\raisebox{0pt}[0pt][0pt]{\rlap{$\vecr{I}$}}\phantom{I}_{\text{v}}}  % TLGF
\newcommand{\tauhat}%
 {\rlap{$\boldsymbol{\hat{\tau}}$}\phantom{\boldsymbol{\tau}}} % Pol. vector
\newcommand{\uhat}{\boldsymbol{\hat{u}}}  % Polarization vector
\newcommand{\what}{\boldsymbol{\hat{w}}}  % Polarization vector
\newcommand{\khat}{\boldsymbol{\hat{k}}}  % Polarization vector
\newcommand{\sigmahat}%
 {\rlap{$\boldsymbol{\hat{\sigma}}$}\phantom{\boldsymbol{\sigma}}} % Pol. vec.


%% \tlsection{x1}{x4}{n} draws a transmission line section
%% from x1 to x4, with nodes at x1 and x4, and labels the 
%% section as number ``n''
\newcommand{\tlsection}[3]{%
                                % Draw top line:
  \mmultido{\rrone=#1+0.3,\rrtwo=#2+-0.3}{1}{%
    \qdisk(#1,0.8){1.5pt} \psline(#1,.8)(\rrone,.8) %
    \psframe(\rrone,0.7)(\rrtwo,0.9) %
    \psline(\rrtwo,0.8)(#2,0.8) \qdisk(#2,0.8){1.5pt} %
                                % Draw bottom line
    \qdisk(#1,-0.8){1.5pt} \psline(#1,-0.8)(\rrone,-0.8) %
    \psframe(\rrone,-0.9)(\rrtwo,-0.7) %
    \psline(\rrtwo,-0.8)(#2,-0.8) \qdisk(#2,-0.8){1.5pt} %
                                % Labels
    \pcline[linestyle=none](\rrone,0)(\rrtwo,0)
    \lput{:U}{\cstack{$Z_{pmn}^{(#3)}$ \\[0.5ex] $\gamma_{mn}^{(#3)}$}}
                                % Dimensions
    \pcline[linewidth=0.5pt]{<->}(\rrone,1.3)(\rrtwo,1.3)
    \pcline{|-|}(\rrone,1.3)(\rrtwo,1.3)
    \lput*{:U}{$h^{(#3)}$}
    }
  }

%% \Ytlsection{x1}{x4}{n} draws a transmission line section
%% from x1 to x4, with nodes at x1 and x4, and labels the 
%% section as number ``n'' using a Y rather than a Z
\newcommand{\Ytlsection}[3]{%
                                % Draw top line:
  \mmultido{\rrone=#1+0.3,\rrtwo=#2+-0.3}{1}{%
    \qdisk(#1,0.8){1.5pt} \psline(#1,.8)(\rrone,.8) %
    \psframe(\rrone,0.7)(\rrtwo,0.9) %
    \psline(\rrtwo,0.8)(#2,0.8) \qdisk(#2,0.8){1.5pt} %
                                % Draw bottom line
    \qdisk(#1,-0.8){1.5pt} \psline(#1,-0.8)(\rrone,-0.8) %
    \psframe(\rrone,-0.9)(\rrtwo,-0.7) %
    \psline(\rrtwo,-0.8)(#2,-0.8) \qdisk(#2,-0.8){1.5pt} %
                                % Labels
    \pcline[linestyle=none](\rrone,0)(\rrtwo,0)
    \lput{:U}{\cstack{$Y_{q}^{(#3)}$ \\[0.5ex] $\gamma_{mn}^{(#3)}$}}
                                % Dimensions
    \pcline[linewidth=0.5pt]{<->}(\rrone,1.3)(\rrtwo,1.3)
    \pcline{|-|}(\rrone,1.3)(\rrtwo,1.3)
    \lput*{:U}{$h^{(#3)}$}
    }
  }


%% \vsource{x} places a voltage generator at location x
\newcommand{\vsource}[1]{%
  \psline(#1,-0.8)(#1,-0.2) \qdisk(#1,-0.8){1.5pt}
  \psline(#1,0.8)(#1,0.2) \qdisk(#1,0.8){1.5pt}
  \pscircle[fillcolor=white](#1,0){0.2}
  %\psline{->}(#1,-0.13)(#1,0.13)
   \rput(#1,0){\renewcommand{\arraystretch}{0}%
     \raisebox{-3pt}[0pt][0pt]{%
     \begin{tabular}{@{}c@{}}
       $\scriptstyle+$ \\ \raisebox{1pt}[0.8\height][0pt]{$\scriptstyle-$}
     \end{tabular}}%
     }
  }



%% \invertedvsource{x} places an upside down voltage generator at location x
\newcommand{\invertedvsource}[1]{%
  \psline(#1,-0.8)(#1,-0.2) \qdisk(#1,-0.8){1.5pt}
  \psline(#1,0.8)(#1,0.2) \qdisk(#1,0.8){1.5pt}
  \pscircle[fillcolor=white](#1,0){0.2}
  %\psline{->}(#1,-0.13)(#1,0.13)
   \rput(#1,0){\renewcommand{\arraystretch}{0}%
     \raisebox{-2pt}[0pt][0pt]{%
     \begin{tabular}{@{}c@{}}
       $\scriptstyle-$ \\ \raisebox{1pt}[0.8\height][0pt]{$\scriptstyle+$}
     \end{tabular}}%
     }
  }



%% \isource{x} places a current generator at location x
\newcommand{\isource}[1]{%
  \psline(#1,-0.8)(#1,-0.2) \qdisk(#1,-0.8){1.5pt}
  \psline(#1,0.8)(#1,0.2) \qdisk(#1,0.8){1.5pt}
  \pscircle[fillcolor=white](#1,0){0.2}
  \psline{->}(#1,-0.13)(#1,0.13)
  }

%% \shuntload{x}{label} places a load at location x with label ``label''
\newcommand{\shuntload}[2]{%
  \psline(#1,-0.8)(#1,0.8) \qdisk(#1,-0.8){1.5pt} \qdisk(#1,0.8){1.5pt}
  \rput*(#1,0){\psframebox[framesep=2pt,boxsep=false]{#2}}
  }
%% \zlabel{x}{label} places label at location x:
\newcommand{\zlabel}[2]{%
  \psline[linestyle=dashed,linewidth=0.6pt](#1,-1.6)(#1,-1)
  \rput[t](#1,-1.7){#2}
  }

\newcommand{\cstack}[1]{\begin{tabular}{@{}c@{}} #1 \end{tabular}}
\newcommand{\khatinc}{\boldsymbol{\hat{k}}^{\text{i}}}
\newcommand{\thetainc}{\theta^{\text{i}}}

%% \seriesload{x1}{x2}{x3}{label} draws a series load 
%% from x1 to x3, centered at x2, 
%% with nodes at x1 and x3, and labels the 
%% load using the third argument:
\newcommand{\seriesload}[4]{%
                                % Draw top line:
  \qdisk(#1,0.8){1.5pt} \psline(#1,.8)(#3,.8) \qdisk(#3,0.8){1.5pt} %
                                % Draw bottom line
  \qdisk(#1,-0.8){1.5pt} \psline(#1,-0.8)(#3,-0.8) \qdisk(#3,-0.8){1.5pt} %
                                % Labels
  \rput*(#2,0.8){\psframebox[boxsep=false]{#4}} %
  }



 %%%%%%%%%%%%%%%%%%%%%% End macro definitions  %%%%%%%%%%%%%%%%%%%%%%%%%%%%%%%
